\section{Reunión 16 de Mayo 2024}

La literatura sobre experiencia de usuario y diseño de páginas web revela la importancia crucial de varios factores en la percepción y comportamiento del usuario. Uno de los hallazgos más significativos es que los usuarios pueden formar una primera impresión sobre el atractivo visual de una página web en tan solo 50 milisegundos, y estas evaluaciones iniciales son consistentes y tienen un fuerte efecto halo, influyendo en percepciones subsecuentes de calidad y usabilidad del sitio \cite{lindgaard2006attention}. Además, la complejidad visual y la estética de una página web están fuertemente correlacionadas; los elementos estructurales como imágenes, enlaces y palabras juegan un papel clave en estas percepciones. Un equilibrio adecuado entre estética y complejidad visual es esencial para diseñar páginas web que sean tanto atractivas como funcionales \cite{michailidou2008visual}. La evaluación heurística ha demostrado ser un método efectivo para identificar problemas de usabilidad, permitiendo a los diseñadores mejorar rápidamente las interfaces basándose en las evaluaciones de un pequeño grupo de evaluadores independientes \cite{nielsen1990heuristic}. En el contexto del aprendizaje multimedia, reducir la carga cognitiva es fundamental; estrategias como la segmentación, el preentrenamiento y la eliminación de redundancias ayudan a minimizar la sobrecarga cognitiva, mejorando así la comprensión y retención del usuario \cite{mayer2003nine}. Finalmente, la experiencia del usuario (UX) no solo depende de la usabilidad y el diseño visual, sino también de factores emocionales como la satisfacción, el enganche y la percepción estética. Estas experiencias emocionales pueden medirse mediante encuestas, entrevistas y análisis de comportamiento, proporcionando una visión integral de cómo los usuarios interactúan con los productos tecnológicos \cite{bargas2011old}.

\subsection{Variables dependientes}
\begin{itemize}
    \item Percepción de Complejidad Visual y Estética: \cite{michailidou2008visual}.
    \begin{itemize}
        \item Definición: La evaluación que los usuarios hacen sobre la complejidad visual y la estética de una página web.
        \item Medición: Clasificación de complejidad visual y estética utilizando escalas de Likert y análisis estadístico para evaluar la percepción de los usuarios. 
        \item Operacionalización: Se les muestra a los entrevistados durante 7 segundos, 20 imágenes de algunas páginas web que fueron parte de las mejores 100 páginas web de UK según Alexa (2007) y se les preguntó a los participantes en una escala likert sobre que tan atestado-limpio, Aburrido-Interesante, Desorganizado-Organizado, Confuso-claro, Feo-bonito, encontraban cada una de las imagenes mostradas. Las variables independientes fueron: menús, imágenes, cantidad de palabras, número de links visibles y Esquinas Superiores Izquierdas.  
        \item Hipostesis: 
        
        H1 The number of menus, images, visible links, words and TLCs a Web page has, is positively related with the page’s level of visual complexity.
        
        H2 User perception of Web page aesthetic characteristics with respect to organization, clearness, cleanliness, interestingness and beautifulness, is linearly related with the number of menus, images, visible links, words and TLCs a Web page has.

        H3 User perception of Web page visual complexity is related with the aesthetic qualities it presents. That is, the more organised, clear, clean, interesting and beautiful a Web page is the less visually complex is perceived as by the user.
    \end{itemize}
    
    \item Evaluaciones de Usabilidad: \cite{nielsen1990heuristic}
    \begin{itemize}
        \item Definición: La facilidad con la que los usuarios pueden utilizar la interfaz para completar tareas específicas.
        \item Medición: Identificación de problemas de usabilidad a través de evaluaciones heurísticas independientes realizadas por múltiples evaluadores, documentando problemas específicos y proponiendo mejoras.
        \item Operacionalización: Se les presentó a los sujetos distintos escenarios (2 print de pantalla de páginas web, una descripción de una página web y una pagina web navegable), los cuales tenía que observar y luego escribir un informe detallado de la usabilidad de dicho escenario, considerando los siguientes criterios \cite{molich1990}: 
        \begin{itemize}
            \item Simple and natural dialogue: Los diálogos no deben contener información irrelevante o raramente necesaria. Cada unidad de información extra en un diálogo compite con las unidades de información relevantes y disminuye su visibilidad relativa. Toda la información debe aparecer en un orden natural y lógico.
            \item Speak the user’s language: El diálogo debe expresarse claramente en palabras, frases y conceptos familiares para el usuario, en lugar de utilizar términos orientados al sistema.
            \item Minimize user memory load: La memoria a corto plazo del usuario es limitada. El usuario no debería tener que recordar información de una parte del diálogo a otra. Las instrucciones para el uso del sistema deben ser visibles o fácilmente recuperables cuando sea apropiado. Las instrucciones complicadas deben simplificarse.
            \item  Be consistent: Los usuarios no deberían tener que preguntarse si diferentes palabras, situaciones o acciones significan lo mismo. Una acción del sistema particular, cuando sea apropiada, siempre debería ser alcanzable mediante una acción específica del usuario. La consistencia también implica coordinación entre subsistemas y entre sistemas independientes importantes con poblaciones de usuarios comunes.
            \item Provide feedback: El sistema siempre debe mantener al usuario informado sobre lo que está ocurriendo, proporcionándole una retroalimentación adecuada dentro de un tiempo razonable.
            \item Provide clearly marked exits: Un sistema nunca debe atrapar a los usuarios en situaciones sin una salida visible. Los usuarios a menudo eligen funciones del sistema por error y necesitarán una "salida de emergencia" claramente marcada para salir del estado no deseado sin tener que pasar por un diálogo extenso.
            \item Provide shortcuts: Las características que hacen que un sistema sea fácil de aprender, como diálogos verbosos y pocos campos de entrada en cada pantalla, a menudo resultan engorrosas para el usuario experimentado. Atajos inteligentes, no visibles para el usuario novato, pueden incluirse en un sistema de manera que satisfaga tanto a los usuarios inexpertos como a los experimentados.
            \item Good error messages: Los buenos mensajes de error son defensivos, precisos y constructivos. Los mensajes de error defensivos atribuyen el problema a deficiencias del sistema y nunca critican al usuario. Los mensajes de error precisos proporcionan al usuario información exacta sobre la causa del problema. Los mensajes de error constructivos ofrecen sugerencias significativas al usuario sobre qué hacer a continuación.
            \item Prevent errors: Incluso mejor que buenos mensajes de error es un diseño cuidadoso que prevenga que ocurra un problema desde el principio.
        \end{itemize}        
        Luego se contaron los errores de usabilidad que cada sujeto encontró en los escenarios y se utilizó como indicador. \textbf{En este paper se evalúa la capacidad y factibilidad de utilizar heurísticas para evaluar diseños web} 
    \end{itemize}
    
    \item Carga Cognitiva y Comprensión: \cite{mayer2003nine}
    \begin{itemize}
        \item Definición: La cantidad de esfuerzo mental que los usuarios deben invertir para procesar la información y la efectividad con la que entienden el contenido.
        \item Medición: Evaluación de la carga cognitiva y comprensión mediante pruebas de transferencia y encuestas post-experimentales para medir el impacto de diferentes estrategias de diseño multimedia.
        \item Operacialización: Se realizaron experimentos comparando distintas formas de presentar la información y luego se les pidió resolver problemas de transferencia. Por ejemplo:
        \begin{itemize}
            \item Efecto de Modalidad para problemas de sobrecarga de canal visual:
            
             Se realizaron seis estudios donde se daban explicaciones científicas presentadas como animación con narración vs animación con texto en pantalla, se encontró mejor rendimiento en las pruebas cuando las palabras se presentaron como narración en lugar de texto en pantalla.
             
             \item Efecto de Segmentación para problemas de sobrecarga de ambos canales (visual y verbal):
             
             Se realizó un estudio en el cual la animación narrada se dividió en 16 segmentos controlados por el alumno. Los estudiantes que recibieron la presentación segmentada tuvieron un mejor rendimiento en las pruebas de transferencia que aquellos que recibieron una presentación continua.
            
             \item Efecto de Preentrenamiento para problemas de sobrecarga de ambos canales (visual y verbal):

             Se realizaron tres estudios en los cuales los estudiantes recibieron un preentrenamiento sobre los nombres y comportamientos de los componentes de los sistemas antes de ver las animaciones narradas.  Los estudiantes con preentrenamiento tuvieron un mejor rendimiento en las pruebas de transferencia que aquellos sin preentrenamiento.

             \item  Efecto de Coherencia para Sobrecarga por Procesamiento Incidental (Esto puede ocurrir cuando se incluye material adicional que no es necesario para comprender el contenido principal, lo que desvía la atención y los recursos cognitivos del estudiante):

             Se realizaron cinco estudios donde se compararon narraciones animadas concisas con narraciones animadas embellecidas con material adicional. Los estudiantes que recibieron narraciones concisas tuvieron un mejor rendimiento en las pruebas de transferencia que aquellos que recibieron narraciones embellecidas.

             \item Efecto de Señalización para Sobrecarga por Procesamiento Incidental: 

             Se realizó un estudio en el cual se añadió señalización (proporcionar pistas sobre cómo procesar el material) a una animación narrada confusa. Los estudiantes que recibieron la versión señalizada de la animación narrada tuvieron un mejor rendimiento en las pruebas de transferencia que aquellos que recibieron la versión no señalizada.

             \item Efecto de Contigüidad Espacial para Sobrecarga por Presentación Confusa (Este tipo de sobrecarga cognitiva ocurre cuando la información esencial se presenta de manera desorganizada, dispersa o redundante, lo que dificulta el procesamiento eficiente por parte del alumno):

             Se realizó un estudio comparando presentaciones integradas (animación con texto en pantalla integrado) con presentaciones separadas (animación con texto en pantalla separado). Los estudiantes que recibieron presentaciones integradas tuvieron un mejor rendimiento en las pruebas de transferencia que aquellos que recibieron presentaciones separadas.

             \item Efecto de Redundancia para Sobrecarga por Presentación Confusa: 

             Se realizaron tres estudios en los cuales se compararon presentaciones redundantes (animación con narración y texto en pantalla) con presentaciones no redundantes (animación con narración únicamente). Los estudiantes que recibieron presentaciones no redundantes tuvieron un mejor rendimiento en las pruebas de transferencia que aquellos que recibieron presentaciones redundantes.

             \item Efecto de Contigüidad Temporal para Sobrecarga por Necesidad de Retención Representacional (ocurre cuando el diseño de la instrucción multimedia obliga al estudiante a mantener información en la memoria de trabajo durante un período prolongado mientras intenta procesar nueva información):

             Se realizaron ocho estudios en los cuales se compararon presentaciones simultáneas (animación y narración presentadas al mismo tiempo) con presentaciones sucesivas (animación completa seguida de narración completa o viceversa). Los estudiantes que recibieron presentaciones simultáneas tuvieron un mejor rendimiento en las pruebas de transferencia que aquellos que recibieron presentaciones sucesivas.

             \item  Efecto de Habilidad Espacial para Sobrecarga por Necesidad de Retención Representacional:

             Se realizaron dos estudios donde se comparó el rendimiento de estudiantes con alta y baja habilidad espacial en presentaciones simultáneas frente a sucesivas. Los estudiantes con alta habilidad espacial se beneficiaron más de las presentaciones simultáneas que los estudiantes con baja habilidad espacial. 
        \end{itemize}
    \end{itemize}

    \item Satisfacción del Usuario: \cite{bargas2011old}
    \begin{itemize}
        \item Definición: El grado de contento o insatisfacción de los usuarios con su experiencia en la página web.
        \item Medición: Encuestas de satisfacción del usuario, entrevistas y análisis de comportamiento.
    \end{itemize}

    \item Enganche y Lealtad del Usuario: \cite{bargas2011old}
    \begin{itemize}
        \item Definición: El nivel de interés y compromiso que los usuarios muestran hacia la página web, y su disposición a regresar o recomendarla.
        \item Medición:  Medidas de tiempo de uso, frecuencia de interacción y tasas de retorno.
    \end{itemize}
\end{itemize}

\subsection{Variables Independientes}
\begin{itemize}
    \item Elementos Estructurales de la Página Web \cite{michailidou2008visual}
    \begin{itemize}
        \item Definición: Incluyen la cantidad y tipo de imágenes, enlaces, palabras, menús y otros componentes visuales y funcionales en una página web.
        \item Medición: Identificación y conteo de elementos visibles en cada página web, utilizando técnicas de análisis del contenido y conteo directo.
        \item Operacionalización: ???
    \end{itemize}
    \item Familiaridad del Usuario con la Página Web
    \begin{itemize}
        \item Definición: Se refiere al grado de conocimiento y experiencia previa del usuario con la página web.
        \item Medición:   Encuestas y cuestionarios donde los participantes indican su nivel de familiaridad con la página web.
    \end{itemize}
    \item Tiempo de Exposición \cite{lindgaard2006attention}
    \begin{itemize}
        \item Definición: El periodo durante el cual los usuarios están expuestos a la página web o a sus elementos visuales.
        \item Medición:  Control experimental del tiempo de visualización.
    \end{itemize}
    \item Complejidad Visual \cite{michailidou2008visual}
    \begin{itemize}
        \item Definición: La cantidad de información visual y el grado de desorden o intrincación en la disposición de los elementos de la página.
        \item Medición: Clasificación de la complejidad visual en escalas de calificación y análisis computacional.
    \end{itemize}
    \item Diseño de la Interfaz (Heurísticas de Usabilidad): \cite{nielsen1990heuristic}
    \begin{itemize}
        \item Definición: La aplicación de principios conocidos de usabilidad (heurísticas) en el diseño de la interfaz.
        \item Medición: Evaluaciones heurísticas realizadas por múltiples evaluadores, documentando problemas de usabilidad específicos.
    \end{itemize}
    \item Estrategias de Diseño Multimedia: \cite{mayer2003nine}
    \begin{itemize}
        \item Definición: Diferentes técnicas utilizadas para presentar información multimedia, como segmentación, preentrenamiento y eliminación de redundancias.
        \item Medición: Experimentos donde se manipulan estas estrategias y se evalúa su impacto en la comprensión y retención del usuario.
    \end{itemize}


\section{Reunión 23 de Mayo 2024}

Meta-análisis (51 publicaciones sobre UX, entre el 2005 y 2009) \cite{bargas2011old}

\begin{itemize}
    \item Productos Estudiados en la Investigación de UX
    \begin{itemize}
        \item  Arte (21\%): Incluye estudios sobre productos como un escritorio de fotografía de audio diseñado para desafiar las concepciones de los usuarios sobre una ciudad.
        \item Aplicaciones y Teléfonos Móviles (21\%): Estudios sobre aplicaciones móviles y teléfonos, clasificándolos en uso de ocio, trabajo o mixto.
        \item Audio, Video, TV (15\%): Estudios que incluyen productos de entretenimiento como aplicaciones de audio y video.
        \item \textbf{Sitios Web (12\%): Investigación sobre la experiencia del usuario con sitios web.}
        \item  Productos Imaginados (9\%): Estudios que se centran en productos teóricos o imaginados.
        \item Juegos Interactivos (6\%): Investigación sobre la experiencia del usuario con juegos interactivos.
        \item Otros Productos (9\%): Incluye diversos productos que no encajan en las categorías anteriores.
    \end{itemize}

    \item Situaciones de Uso en la Investigación de UX
    \begin{itemize}
        \item Situaciones de Uso Abiertas (61\%): Donde se proporcionan instrucciones generales sin detallar los pasos específicos.
        \item Tareas Controladas (33\%): Donde se definen tareas precisas que los participantes deben completar.
        \item Uso Iniciado por el Usuario (20\%): Donde los usuarios son libres de elegir si, cuándo y cómo utilizan los productos.
    \end{itemize}

    \item Dimensiones de la Experiencia en la Investigación de UX
    \begin{itemize}
        \item UX Genérica (41\%): Entrevistas y grupos focales que no especifican las dimensiones exactas de UX evaluadas.
        \item  \textbf{Afecto y Emoción (24\%): Medido con herramientas como la escala SAM}.
        \item Disfrute y Diversión (17\%): Evaluado mediante categorías de juego y respuestas emocionales.
        \item  Estética y Atractivo (15\%): Evaluado con cuestionarios sobre estética visual.
        \item Calidad Hedónica (14\%): Medido con herramientas como AttrakDiff.
        \item Compromiso y Flujo (12\%): Evaluado con entrevistas y escalas como FSS.
        \item Motivación (8\%): Evaluado mediante métodos como los probes.
        \item Encantamiento (6\%): Entrevistas para entender el encanto de la tecnología.
        \item Frustración (5\%): Evaluado mediante entrevistas semi-estructuradas.
    \end{itemize}

    \item \textbf{Métodos de Recopilación de Datos de UX}
    \begin{itemize}
        \item \textbf{Cuestionarios (53\%): El método más utilizado para evaluar UX.}
        \item \textbf{Entrevistas Semi-Estructuradas (20\%): Utilizadas para obtener feedback detallado de los usuarios.}
        \item  \textbf{Observación de Usuarios (17\%): Observación in situ del uso de productos.}
        \item \textbf{Grabaciones de Video (17\%): Utilizadas para capturar interacciones.}
        \item Grupos Focales (15\%): Discusiones grupales para investigar preferencias y experiencias.
        \item  Entrevistas Abiertas (12\%): Utilizadas para obtener una comprensión profunda de la experiencia del usuario.
        \item Diarios (11\%): Usados para evaluar emociones a través de registros diarios.
        \item Probes (9\%): Kits de probes entregados a los participantes para capturar experiencias personales.
        \item  Collages o Dibujos (8\%): Utilizados para que los participantes articulen sus experiencias.
        \item Fotografías (8\%): Los usuarios toman fotos relacionadas con atributos en su entorno.
        \item \textbf{Mediciones Psicofisiológicas (5\%): Medición de respuestas fisiológicas durante la interacción con el producto.}
    \end{itemize}  
\end{itemize}
    
\end{itemize}
