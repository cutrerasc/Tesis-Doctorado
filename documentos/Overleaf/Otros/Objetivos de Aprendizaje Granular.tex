\newpage
\section{Objetivos de aprendizaje Granulares para pensiones privadas.}

En esta sección se intenta realizar un programa de curso para el proceso Jubilatorio por lo que se detallan todos los contenidos que se espera que sean abordados por este curso. 
\vspace{0.2cm}
\begin{enumerate}
    \item \textbf{Identificar y entender las diferentes modalidades de pensión (como Renta Vitalicia Inmediata, Retiro Programado, etc.) que se pueden seleccionar en el formulario SCOMP.} 
        \begin{itemize}
            \item Qué es una pensión 
            \item Qué es el proceso de SCOMP 
            \item Certificado de Saldo 
            \item Proceso de Solicitud de Pensión 
        \end{itemize}    
        \textbf{Objetivos específicos: }
        \begin{enumerate}
            \item Comprender en qué consiste la modalidad de \textit{Renta Vitalicia Inmediata}, incluyendo sus condiciones especiales de cobertura, como periodos garantizados y aumento temporal de pensión.
            \begin{itemize}
                \item Qué es una compañía de seguros y cómo funcionan.
                \item Qué es la modalidad de Renta Vitalicia y cómo funciona 
                \begin{itemize}
                    \item Propiedad de los fondos
                \end{itemize}
                \item Beneficiarios Legales y su rol en esta modalidad
                \item Condiciones especiales de cobertura
                \begin{itemize}
                    \item Aumento temporal de pensión (Periodos escalonados) 
                    \item Periodos garantizados de sobrevivencia 
                    \item Clausula de incremento de porcentaje 
                \end{itemize}
            \end{itemize}
    
        \item Entender cómo funciona la modalidad de Retiro Programado, incluyendo los factores que influyen en los montos y sus cambios a lo largo del tiempo.
        \begin{itemize}
        \item Qué es una AFP y cómo funcionan 
        \item Qué es la modalidad de Retiro programado y cómo se calcula el monto de pensión anual 
        \begin{itemize}
        \item Saldo de Cuenta Individual 
        \item Expectativa de Vida (estimación)
        \item Beneficiarios legales (Rol)
        \item Tasa de interés técnica (riesgo) 
        \end{itemize}
        \item Propiedad de los fondos 
        \item Cambio de Modalidad (Requisitos) 
        \item Herencia 
        \end{itemize}

        \item Entender cómo funciona la modalidad de Renta Temporal con Renta Vitalicia Diferida, incluyendo el periodo de diferimiento, y las condiciones que afectan los montos de pensión.  
        \begin{itemize}
            \item Renta temporal 
            \item Renta Vitalicia 
            \begin{itemize}
                \item Condiciones especiales de cobertura 
            \end{itemize}
            \item Cómo funciona la modalidad 
            \begin{itemize}
                \item Requisitos para optar a la modalidad
                \item División de fondos entre AFP y CSV 
                \item Periodos de diferimiento 
                \item Coberturas o beneficios que aportan cada parte de la modalidad.
            \end{itemize}
        \end{itemize}
        \item Entender la modalidad combinada de Renta Vitalicia Inmediata con Retiro Programado y las implicancias para el pensionado
        \begin{itemize}
            \item Retiro Programado
            \item Renta Vitalicia 
            \begin{itemize}
                \item Condiciones especiales 
            \end{itemize}
            \item Cómo funciona la modalidad 
            \begin{itemize}
                \item División de Fondos entre AFP y CSV
                \item Requisitos para optar a la modalidad 
                \item Coberturas o beneficios que aportan cada parte de la modalidad.
            \end{itemize}
        \end{itemize}
        \end{enumerate}
    

\item \textbf{Entender cómo las características individuales o de grupo familiar influyen sobre las preferencias de elegir alguna de las distintas modalidades de pensión. (Pensión de sobrevivencia y herencia)}

\begin{itemize}
    \item Beneficiarios Legales: 
    \begin{itemize}
        \item Qué es un beneficiario legal
        \item Tipos de beneficiarios: Cónyuge, hijos, conviviente civil, etc.
        \item Cómo afecta el cálculo de la modalidad de Periodo Garantizado a una Renta Vitalicia.  
        \item Cómo afecta el monto final de una pensión por Retiro Programado. 
    \end{itemize}
    \item Pensión de Sobrevivencia
    \begin{itemize}
        \item Qué es una pensión de Sobrevivencia 
        \item Requisitos de los beneficiarios para acceder 
        \item Cómo se calcula el Monto de Pensión de sobrevivencia 
    \end{itemize}
    \item Herencia: 
    \begin{itemize}
        \item Diferencia entre CSV y AFP (Propiedad de los Fondos) 
        \item Qué es la herencia 
        \item Quiénes pueden recibirla 
        \item Cómo se calculan los montos 
    \end{itemize}
    \item Diferencia entre Pensión de Sobrevivencia y herencia 
\end{itemize}

\item \textbf{Comprender las principales diferencias entre las modalidades de pensión.}
\begin{itemize}
    \item Propiedad y administración de los fondos
    \begin{itemize}
        \item Diferencias entre AFP y aseguradoras como entidades administradoras.
        \item Qué significa que los fondos se mantengan en la cuenta individual (Retiro Programado).
        \item Qué implica transferir los fondos a una aseguradora (Renta Vitalicia).
        \item Consecuencias en términos de herencia y control del dinero.
    \end{itemize}
    \item Forma de cálculo del monto de pensión
    \begin{itemize}
        \item Cómo se calcula el Retiro Programado: saldo, expectativa de vida y tasa de interés técnica.
        \item Cómo se determina el monto fijo en una Renta Vitalicia Inmediata.
        \item Estructura de cálculo en las rentas combinadas (temporal + vitalicia diferida, o vitalicia + retiro).
        \item Factores que hacen variar o fijar el monto de pensión.
    \end{itemize}
    \item Estabilidad del monto mensual
    \begin{itemize}
        \item Comportamiento del Retiro Programado a lo largo del tiempo: por qué disminuye.
        \item Estabilidad del monto en Renta Vitalicia.
        \item Combinaciones: cuándo se obtiene mayor pensión al inicio y cómo disminuye después.
        \item Riesgos y beneficios de cada estructura de pagos.
    \end{itemize}
    \item Cobertura en caso de fallecimiento
    \begin{itemize}
        \item Pensión de Sobrevivencia en el Retiro programado. 
        \item Herencia en el Retiro Programado
        \item Pensión de Sobrevivencia en la Renta Vitalicia (sin y con periodo garantizado).
    \end{itemize}
    \item Comisiones y costos asociados
    \begin{itemize}
        \item Comisiones cobradas por las AFP en el Retiro Programado.
        \item Comisiones cobradas por aseguradoras en Renta Vitalicia (incluyendo intermediarios).
        \item Cómo afecta el uso de asesores previsionales o agentes de venta.
        \item Impacto de los costos en el monto de pensión ofrecido.
    \end{itemize}
    \item Flexibilidad para cambiar de modalidad
    \begin{itemize}
        \item Posibilidad de cambiar de Retiro Programado a otra modalidad.
        \item Irreversibilidad de las Rentas Vitalicias una vez contratadas.
    \end{itemize}
    \item Requisitos para optar a cada modalidad
    \begin{itemize}
        \item Requisitos técnicos y legales para elegir Renta Vitalicia (saldo mínimo, existencia de ofertas).
        \item Disponibilidad automática de Retiro Programado como opción por defecto.
        \item Requisitos adicionales para acceder a Rentas mixtas o combinadas.
        \item Cómo el perfil del usuario (edad, saldo, beneficiarios) condiciona la disponibilidad real de las opciones.
    \end{itemize}
    \item Condiciones especiales de cobertura
    \begin{itemize}
        \item Qué es el aumento temporal de pensión y en qué contextos aplica.
        \item Qué es un periodo garantizado de sobrevivencia y por cuánto tiempo puede cubrir.
        \item Qué es la cláusula de incremento de pensión.
        \item Costos y beneficios de añadir estas coberturas opcionales.
    \end{itemize}
    \item Comparación práctica entre modalidades
    \begin{itemize}
        \item Tabla resumen: propiedad de fondos, estabilidad, herencia, comisiones, cambio, requisitos.
        %\item Casos tipo: ¿qué elegiría una persona sin beneficiarios? ¿con hijos menores? ¿con alto saldo?
        %\item Ejercicios de simulación: elegir la mejor opción para distintos perfiles.
    \end{itemize}
\end{itemize}

\item \textbf{Entender los conceptos que se utilizan al momento de entrar al SCOMP }
\begin{itemize}
    \item Certificado de Saldo
    \begin{itemize}
        \item Qué es
        \item Cómo se obtiene
        \item Para qué se obtiene 
        \item Vigencia
    \end{itemize}
    \item SCOMP (Sistema de Consultas y Ofertas de Montos de Pensión) 
    \begin{itemize}
        \item Qué es, para qué sirve y cómo funciona
        \item Formulario de Solicitud de Ofertas
        \begin{itemize}
            \item Quién lo ingresa
            \item Cantidad de Consultas
            \item Modalidades y clausulas especiales
        \end{itemize}
        \item Certificado de Ofertas
        \begin{itemize}
            \item Qué es 
            \item Vigencia 
        \end{itemize}
    \end{itemize}
    \item Ofertas Externas y Remate
    \begin{itemize}
        \item Qué es una oferta externa
        \item En qué consiste el proceso de Remate   
    \end{itemize}
    \item Finalización del proceso
    \begin{itemize}
        \item Cómo aceptar una oferta y dónde realizar el trámite 
        \item Materialización de la elección 
        \begin{itemize}
            \item Formulario “Selección de Modalidad de Pensión” 
        \end{itemize}
        \item Causas de Finalización de un proceso 

    \end{itemize}
\end{itemize}



\item \textbf{Conocer el proceso completo de la solicitud de pensiones}
\begin{itemize}
    \item Inicio del proceso en la AFP
    \begin{itemize}
        \item Presentación de la Solicitud de Pensión y Declaración de Beneficiarios
        \item Documentación necesaria
    \end{itemize}
    \item Emisión del Certificado de Saldo
    \begin{itemize}
        \item Contenido del certificado: saldo acumulado, datos personales, beneficiarios
        \item Vigencia legal y uso como prerequisito para ingresar ofertas
    \end{itemize}
    \item Solicitud de ofertas en SCOMP
    \begin{itemize}
        \item Llenado de la Solicitud de Ofertas: modalidades deseadas, condiciones especiales
        \item Dónde realizar la solicitud: en AFP, aseguradora o con asesor (online o presencial)
    \end{itemize}
    \item Recepción del Certificado de Ofertas
    \begin{itemize}
        \item Plazos de entrega: dentro de 4–8 días hábiles
        \item Vigencia del certificado: 12 días hábiles
        \item Contenido: ofertas de Retiro Programado y Rentas Vitalicias, incluyendo comisiones, tasas de riesgo, montos brutos o netos. 
    \end{itemize}
    \item Evaluación de ofertas
    \begin{itemize}
        \item Opciones: aceptar, cotizar nuevamente (hasta 3 veces con el mismo certificado), solicitar oferta externa o remate, desistir
    \end{itemize}
    \item Remate de pensión
    \begin{itemize}
        \item Cómo funciona el remate de una pensión
    \end{itemize}
    \item Selección y aceptación de modalidad
    \begin{itemize}
        \item Pasos para seleccionar y aceptar una modalidad 
        \begin{itemize}
        \item Firma de “Aceptación de Oferta”
        \item Traslado a AFP o CSV según corresponda
        \item Entrega del formulario “Selección de Modalidad” y documentos asociados: certificado de ofertas, oferta externa (si aplica)
        \end{itemize}
    \end{itemize}
    \item Traspaso de fondos e inicio de pagos
    \item Desistimiento y plazos legales
%    \item Herramientas y canales disponibles
\end{itemize}

\item \textbf{Entender las principales causales que generan incertidumbre o molestia (Ejemplo qué pasa si los fondos de AFP son muy pequeños y se acaban)	}	
\begin{itemize}
    \item Demora en la emisión de certificados y ofertas
    \begin{itemize}
        \item Tiempos de espera
        \item Consecuencias de los plazos incumplidos: necesidad de reiniciar el proceso.
    \end{itemize}
    \item Errores o discrepancias en datos declarados en los formularios
    \item Complejidad del proceso (Largo, confuso y demasiadas decisiones que tomar) 
    \item Cambios en montos o condiciones inesperadas (Valor cuota, variaciones del mercado) 
    \item Falta de claridad sobre comisiones
    \item Otras causales de reclamo
    \begin{itemize}
        \item Demora o incumplimiento en pagos de pensión
        \item Errores en inclusión/exclusión de beneficiarios.
        \item Disconformidad con montos calculados
    \end{itemize}
    \item Irreversibilidad de decisiones    
\end{itemize}

\item \textbf{Entender elementos complementarios para la toma de decisiones}
\begin{enumerate}
    \item Excedentes de libre disposición
    \begin{itemize}
        \item ¿Qué es el Excedente de Libre Disposición?
        \item Requisitos para poder retirar ELD
        \item Cálculo del monto del ELD
        \item Procedimiento para retirar el excedente
        \item Opciones de retiro y tratamiento tributario
        \item Impacto en pensión mensual
        \item Documentación y solicitud
    \end{itemize}
    \item Otros Ahorros Ahorro previsional Voluntario APV
    \begin{itemize}
        \item Definición y características de APV
        \begin{itemize}
        \item Tipos de APV
        \begin{itemize}
            \item Cotizaciones voluntarias (exclusivas AFP).
            \item Depósitos APV.
            \item Depósitos convenidos 
        \end{itemize}
        \end{itemize}
        \item APVC (Ahorro Previsional Voluntario Colectivo)
        \begin{itemize}
            \item ¿Qué es?
            \item ¿Cómo funciona?
            \item Requisitos y Condiciones
            \item Traspasos y finalización de vinculaciones laborales
        \end{itemize}
        \item Cuenta 2 (Cuenta de Ahorro Voluntario AFP)
        \begin{itemize}
            \item Definición y propósito
            \item Flexibilidad: hasta 24 giros al año; fondos disponibles para distintos fines 
            \item Incentivos al destinar a pensión: Estado otorga beneficios tributarios y bonos
            \item Diferencias frente al APV: liquidez, objetivos, sin beneficios tributarios automáticos
        \end{itemize}
        \item Entidades administradoras y regulación
        \item Régimen tributario y beneficios
       % \begin{itemize}
            %\item Topes de deducción
            %\item Modos de tributación
            %\item Retiro anticipado
        %\end{itemize}
        \item Movilidad de Fondos
        \item Conexión con la pensión 
        %\item Costos y precauciones 
    \end{itemize}
\end{enumerate}
\end{enumerate}
