\newpage
\section*{Perfiles de Usuario}
\subsection*{Perfil de Ingreso}
Para cualquier persona que ingrese a la secuencia de aprendizaje, son necesarios los siguientes requisitos mínimos: 
\begin{itemize}
    \item Comprensión básica del sistema Previsional (Existen AFP, se ahorra mensualmente para la jubilación, etc.) 
    \item Conocimientos mínimos financieros (Ingresos, ahorro, UF, etc.)
    \item Habilidad de comprensión lectora básica (interpretar y entender textos breves)
    \item Capacidad de mantener atención por periodos de 10 a 15 minutos aproximadamente
    \item Capacidad para interactuar con pantallas táctiles, botones o enlaces simples.
\end{itemize}

   
De lo anterior se desprende que: Todos aquellos usuarios que no posean alguno de estos requisitos necesitarán una atención más personalizada de la mano de algún cercano o bien de atención a público de los organismos correspondientes. 

Para aquellos usuarios que sí cumplen estos requisitos mínimos, se definieron 16 perfiles de usuario de acuerdo con los siguientes criterios que aparecen en la tabla \ref{tab:perfiles}:
\begin{itemize}
    \item Tiene derecho a SCOMP: Si o no. (Condición económica)
    \item Tiene beneficiarios legales: Si o no (Condición familiar) 
    \item Tiene experiencia previa con el sistema de pensiones o alguna otra institución financiera: si o no (Condición de conocimiento previo) 
    \item Tiene una alta capacidad de comprensión y/o aprendizaje: Si o no. (Condición de capacidades cognitivas). 
\end{itemize}


\begin{table}
    \centering
    \scriptsize
    \begin{tabular}{|p{1cm}|p{1cm}|p{1.5cm}|p{1.5cm}|p{1.5cm}|p{6cm}|}
    \hline
        Perfil & SCOMP & Beneficiarios & Comprensión y/o aprendizaje & Experiencia & Supuesto\\
        \hline
         1& X & X & X & X & Asimila rápidamente terminología técnica y domina el proceso de solicitud de pensión, así como las modalidades de pensión y otros conceptos financieros relevantes.\\
         \hline
         2& X & X & X & & Al no tener experiencia previa en proceso de pensión y/o otras instituciones financieras, es necesario mostrar información introductoria.\\
         \hline
         3& X & X &  & X & Al tener experiencia previa, pero dificultades en la comprensión, podría ser necesario utilizar recursos simples, con información concisa.  \\
         \hline
         4& X & X &  & & No entra al sistema\\
         \hline
         5& X &  & X & X  & Asimila rápidamente terminología técnica y domina el proceso de solicitud de pensión, así como las modalidades de pensión y otros conceptos financieros relevantes. Queda fuera la información sobre beneficiarios legales.\\
         \hline
         6& X &  & X & & Al no tener experiencia previa en proceso de pensión y/o otras instituciones financieras, es necesario mostrar información introductoria. Queda fuera la información sobre beneficiarios legales.\\
         \hline
         7& X &  &  & X & Al tener experiencia previa, pero dificultades en la comprensión, podría ser necesario utilizar recursos simples, con información concisa. Queda fuera la información sobre beneficiarios legales.  \\
         \hline
         8& X &  &  & & Queda fuera del sistema\\
         \hline
         9&  & X & X & X & Asimila rápidamente terminología técnica y domina el proceso de solicitud de pensión, así como las modalidades de pensión y otros conceptos financieros relevantes. Queda fuera la información sobre SCOMP.\\
         \hline
         10& & X & X & & Al no tener experiencia previa en proceso de pensión y/o otras instituciones financieras, es necesario mostrar información introductoria. Queda fuera la información sobre SCOMP. \\ \hline
         11& & X & & X & Al tener experiencia previa, pero dificultades en la comprensión, podría ser necesario utilizar recursos simples, con información concisa. Queda fuera la información sobre SCOMP. \\
         \hline
         12& & X & & & No entra al sistema \\ 
         \hline
         13 & & & X& X & Asimila rápidamente terminología técnica y domina el proceso de solicitud de pensión, así como las modalidades de pensión y otros conceptos financieros relevantes. Queda fuera la información sobre BL y SCOMP. (perfil poco probable)\\
         \hline
         14 & & &X& & Al no tener experiencia previa en proceso de pensión y/o otras instituciones financieras, es necesario mostrar información introductoria. Queda fuera la información sobre BL y SCOMP.\\
         \hline
         15 & & & & X& Al tener experiencia previa, pero dificultades en la comprensión, podría ser necesario utilizar recursos simples, con información concisa. Queda fuera la información sobre BL y SCOMP.\\
         \hline
         16 & & & & & No entra al sistema.\\
         \hline
    \end{tabular}
    \caption{Perfiles de Usuario}
    \label{tab:perfiles}
\end{table}

Con esta tabla es posible determinar las similitudes y diferencias de los diferentes perfiles, así reordenando y simplificando la tabla anterior, se toman las siguientes decisiones: 

\begin{itemize}
       \item Se generan solo 4 perfiles generales de acuerdo con los criterios de ``Capacidad de comprensión" y ``Experiencia previa". Es decir afectan el tipo de recurso de aprendizaje que cada quien necesitará. Estos son:  
    \begin{enumerate}
        \item Alta comprensión y alta experiencia: Usuarios que asimilan rápidamente nueva información y dominan el proceso de solicitud de pensión, así como las modalidades de pensión y otros conceptos financieros relevantes para dicho proceso.
        \item Alta comprensión y baja experiencia: Al no tener experiencia previa en el proceso de pensión y/o otras instituciones financieras, es necesario mostrar información introductoria básica, sin embargo al poseer un alto nivel de comprensión, podría adquirir y dominar rápidamente estos nuevos conceptos.
        \item Baja comprensión y alta experiencia:  Usuarios que probablemente ya vivieron algún otro proceso de pensión previo, por ejemplo la de algún conocido o cercano, sin embargo, es probable que los conocimientos no los haya consolidado de buena manera, ni tampoco conozca todos los elementos necesarios para tomar una decisión, podría ser necesario utilizar recursos simples, con información concisa para re-familiarizarlos con el proceso.
        \item Baja comprensión y baja experiencia: Se eliminan del sistema de aprendizaje, se espera que estas personas puedan recibir el apoyo de otras personas en su entorno, ya sean familiares o personas de atención al público de los diferentes organismos, en una primera aproximación al proceso de pensión. 
    \end{enumerate}

    \item Además se generan otros 4 perfiles de acuerdo con los criterios de ``Tiene derecho a SCOMP", lo cual lo determina el nivel de ahorro del usuario y ``Tiene beneficiarios legales (BL)", determinado por la composición familiar. Estos criterios dan cuenta de los contenidos que serán necesarios abordar para cada perfil de persona. Estos son: 
    \begin{enumerate}
        \item Tiene derecho a SCOMP y tiene BL: Es el usuario que más contenidos debe dominar, ya que necesita saber como rellenar el formulario SCOMP y como afectan sus beneficiarios legales a su posible pensión, ya sea por retiro programado, renta vitaliacia o alguna modalidad combinada.   
        \item Tiene derecho a SCOMP y no tiene BL: Debe ser capaz de rellenar el formulario SCOMP, podría no ser necesario mostrar contenido sobre BL.  
        \item No tiene derecho a SCOMP y si tiene BL: Debe comprender como afectan a su posible pensión (Retiro Programado) el tener beneficiarios legales. 
        \item No tiene derecho a SCOMP ni BL: Posiblemente es el usuario que menos contenidos necesita aprender. 
    \end{enumerate}
    
\end{itemize}

La tabla que resume esta información, junto con el tipo de contenido y secuencia que deberían ser considerados para cada perfil, se detalla a continuación. 

\begin{table}[ht]
\centering
\begin{tabular}{|p{2cm}|p{4cm}|p{4cm}|p{4cm}|p{2cm}|}
\hline
& \textbf{Comprensión y experiencia} 
  & \textbf{Comprensión sin experiencia} 
    & \textbf{Sin comprensión y con experiencia} 
      & \textbf{Sin comprensión y sin experiencia} \\ 
\hline

\textbf{Con SCOMP y BL} 
  & \begin{minipage}[t]{\linewidth}
    - Tipo de contenido: Sin restricción en complejidad\\
    - Flexibilidad en el orden de presentación (supuesto)\\
    - Incluye preferencias 
    \end{minipage}
  & \begin{minipage}[t]{\linewidth}
    - Tipo de contenido: Sin restricción en complejidad \\
    - Orden de presentación estructurado. 
    - No incluye preferencias 
    \end{minipage}
  & \begin{minipage}[t]{\linewidth}
    - Tipo de contenido: Con restricción en complejidad\\
    - Flexibilidad en el orden de presentación\\
    - Incluye preferencias
    \end{minipage}
  & No entra al sistema, requiere apoyo de un tercero. \\
\hline

\textbf{Con SCOMP s/BL} 
  & \begin{minipage}[t]{\linewidth}
    - Tipo de contenido: Sin restricción en complejidad\\
    - Flexibilidad en el orden de presentación (supuesto)\\
    - Incluye preferencias 
    \end{minipage}
  & \begin{minipage}[t]{\linewidth}
    - Tipo de contenido: Sin restricción en complejidad \\
    - Orden de presentación estructurado. 
    - No incluye preferencias 
    \end{minipage}
  & \begin{minipage}[t]{\linewidth}
    - Tipo de contenido: Con restricción en complejidad\\
    - Flexibilidad en el orden de presentación\\
    - Incluye preferencias
    \end{minipage}
  & No entra al sistema, requiere apoyo de un tercero. \\
\hline

\textbf{Sin SCOMP c/BL Retiros Programados}
  & \begin{minipage}[t]{\linewidth}
    - Tipo de contenido: Sin restricción en complejidad\\
    - Flexibilidad en el orden de presentación (supuesto)\\
    - Incluye preferencias 
    \end{minipage}
  & \begin{minipage}[t]{\linewidth}
    - Tipo de contenido: Sin restricción en complejidad \\
    - Orden de presentación estructurado. 
    - No incluye preferencias 
    \end{minipage}
  & \begin{minipage}[t]{\linewidth}
    - Tipo de contenido: Con restricción en complejidad\\
    - Flexibilidad en el orden de presentación\\
    - Incluye preferencias
    \end{minipage}
  & No entra al sistema, requiere apoyo de un tercero.\\
\hline

\textbf{Sin SCOMP s/BL} 
  & \begin{minipage}[t]{\linewidth}
    - Tipo de contenido: Sin restricción en complejidad\\
    - Flexibilidad en el orden de presentación (supuesto)\\
    - Incluye preferencias 
    \end{minipage}
  & \begin{minipage}[t]{\linewidth}
    - Tipo de contenido: Sin restricción en complejidad \\
    - Orden de presentación estructurado. 
    - No incluye preferencias 
    \end{minipage}
  & \begin{minipage}[t]{\linewidth}
    - Tipo de contenido: Con restricción en complejidad\\
    - Flexibilidad en el orden de presentación\\
    - Incluye preferencias
    \end{minipage}
  & No entra al sistema, requiere apoyo de un tercero. \\
\hline

\end{tabular}
\caption{Tabla de perfiles según SCOMP/BL y niveles de comprensión-experiencia.}
\label{tab:perfiles_scomp_bl}
\end{table}


\subsection*{Perfil de Egreso}

Para los perfiles de egreso se considerarán 2 criterios: nivel de ahorro (sin ahorro, sin derecho a SCOMP y con derecho a SCOMP) y composición familiar (con beneficiarios legales y sin beneficiarios legales). Donde estos determinan los contenidos mínimos que debe conocer el usuario al terminar la secuencia de aprendizaje. 


\begin{table}[H]
\centering
\begin{tabular}{|p{3cm}|p{6cm}|p{6cm}|}
\hline
& \textbf{Con BL} & \textbf{Sin BL} \\ 
\hline

\textbf{Sin ahorro AFP}
  & No entra
  & No entra
\\ 
\hline

\textbf{Sin SCOMP}
  & \begin{minipage}[t]{\linewidth}
  Retiros Programados:\\
    \quad– Cambio de AFP\\
    \quad– Recálculo anual\\
    Beneficiarios:\\
    \quad– ¿Quienes son?\\
    \quad– Montos\\
    \quad– Tiempo (hasta cuando)
    \end{minipage}
  & \begin{minipage}[t]{\linewidth}
  Retiros Programados:\\
    \quad– Cambio de AFP\\
    \quad– Recálculo anual\\
    Beneficiarios:\\
    \quad– ¿Quienes son?\\
    Herencia:\\
    \quad– ¿Quienes pueden heredar?\\
    \quad– Cuánta flexibilidad (Ley de herencias) 
    \end{minipage}
\vspace{0.1cm}\\
\hline

\textbf{Con SCOMP}
  & \begin{minipage}[t]{\linewidth} Retiros Programados:\\
    \quad– Cambio de AFP\\
    \quad– Recálculo anual\\
    Rentas vitalicias:\\
    \quad– Cláusulas especiales\\
    Modalidades mixtas\\
    Comparador de modalidades\\
    Beneficiarios
    \end{minipage}
  & \begin{minipage}[t]{\linewidth} Retiros Programados:\\
    \quad– Cambio de AFP\\
    \quad– Recálculo anual\\
    Rentas vitalicias:\\
    \quad– Cláusulas especiales\\
    Modalidades mixtas\\
    Comparador de modalidades\\
    Beneficiarios:\\
    \quad– Quienes son y cómo acceder si es que estos no son los estipulados por la ley.
    \end{minipage}
    \vspace{0.1cm}
\\ 
\hline
\end{tabular}
\caption{Tabla de contenidos según SCOMP y presencia de BL}
\label{tab:ejemplo_scomp_bl}
\end{table}


\subsection*{Perfilamiento de los usuarios} 
\textit{Pensando en un posible experimento}

Para poder perfilar a los usuarios y saber a cuál de estos grupos pertenecen, se establecen las siguientes preguntas: 

\begin{enumerate}
    \item \textbf{Derecho a SCOMP}
    \begin{itemize}
        \item ¿Cuál es, aproximadamente su nivel de ahorro en la AFP? 
    \end{itemize}
    \item \textbf{Existencia de Beneficiarios Legales}
    \begin{itemize}
        \item Actualmente ¿Se encuentra casada/o o mantiene un acuerdo de unión civil?
        \item ¿Tiene hijos menores de 24 años que se encuentren estudiando? (si su respuesta fue si, ¿cuántos?) 
    \end{itemize}
    \item \textbf{Capacidad de comprensión y conocimiento financiero}
    \begin{itemize}
        \item Capacidad de comprensión financiera (Preguntas sobre interés compuesto) 
        \item Capacidad de comprensión lectora (Investigar algún tipo de test probado en la literatura, por ejemplo: test de Cloze) 
    \end{itemize}
    \item \textbf{Experiencia previa con el sistema de pensiones}
    \begin{itemize}
        \item ¿Ha recibido asesoría Previsional alguna vez / en los últimos X años? 
        \item ¿Ha acompañado a alguien en el proceso de jubilación alguna vez /en los últimos X años? 
    \end{itemize}
\end{enumerate}
    

