\documentclass{article}
\usepackage{graphicx} % Required for inserting images
% Language setting
% Replace `english' with e.g. `spanish' to change the document language
\usepackage[english]{babel}
\usepackage[utf8]{inputenc}

% Set page size and margins
% Replace `letterpaper' with `a4paper' for UK/EU standard size
\usepackage[letterpaper,top=2cm,bottom=2cm,left=3cm,right=3cm,marginparwidth=1.75cm]{geometry}
\usepackage{pgfgantt}
\usepackage{pdflscape}

% Useful packages
\usepackage{amsmath}
\usepackage{graphicx}
%\usepackage[colorlinks=true, allcolors=blue]{hyperref}
\usepackage{comment}
\usepackage{etoolbox}
\apptocmd{\sloppy}{\hbadness 10000\relax}{}{}

% To make cites and references
%\usepackage{natbib}
\usepackage[hidelinks,pdfusetitle,pdfdisplaydoctitle]{hyperref}
\usepackage[notocbib]{apacite} 
\usepackage{doi}
\renewcommand{\doitext}{}

%Para las tablas 
\usepackage{array}
\usepackage{longtable}
%\usepackage{nopageno}
\date{}
\title{Formato proyecto de tesis}
\begin{document}
\maketitle

\begin{itemize}
    \item Portada:
    \begin{itemize}
        \item Título de su Proyecto de Tesis
        \item Nombre del Alumno (Investigador Responsable)
        \item Nombre del Tutor de Tesis y, si lo hubiera, el Cotutor de Tesis (ver más abajo detalles sobre el Comité de Tesis)
    \end{itemize}
    \item Resumen: En esta sección, resuman de manera clara y concisa los puntos clave de su proyecto, incluyendo los objetivos, la metodología y los resultados esperados.
    \item Investigación Propuesta:
    \begin{itemize}
        \item Marco Teórico y Discusión Bibliográfica: Expliquen el contexto del problema y la hipótesis que han planteado.
        \item Hipótesis del Trabajo de Investigación: Describan la hipótesis central de su proyecto.
        \item Objetivos Generales y Específicos: Definan los objetivos que permitirán poner a prueba la hipótesis propuesta.
        \item Metodología: Detallen la metodología que utilizarán en su investigación.
        \item Plan de Trabajo: Se sugiere presentar un plan de trabajo con una propuesta de carta Gantt para la organización de las actividades.
        \item Trabajo por Adelantado o Resultados Preliminares: Si han avanzado en su investigación, presenten los resultados preliminares que hayan obtenido.
    \end{itemize}
\end{itemize}












Informe Escrito: Deben entregar este informe escrito al Comité de Tesis con anticipación a la presentación oral para su revisión y corrección.

\end{document}