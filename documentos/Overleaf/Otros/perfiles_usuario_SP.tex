%\section*{Perfiles de Usuario}

\renewcommand{\arraystretch}{1.5}

Para cualquier persona que ingrese a la secuencia de aprendizaje, son necesarios los siguientes requisitos mínimos: 
\begin{itemize}
    \item Comprensión básica del sistema Previsional (Existen AFP, se ahorra mensualmente para la jubilación, etc.) 
    \item Conocimientos mínimos financieros (Ingresos, ahorro, UF, etc.)
    \item Habilidad de comprensión lectora básica (interpretar y entender textos breves)
    \item Capacidad de mantener atención por periodos de 10 a 15 minutos aproximadamente
    \item Capacidad para interactuar con pantallas táctiles, botones o enlaces simples.
\end{itemize}

De lo anterior se desprende que: Todos aquellos usuarios que no posean alguno de estos requisitos necesitarán una atención más personalizada de la mano de algún cercano o bien de atención a público de los organismos correspondientes. 

Para aquellos usuarios que sí cumplen estos requisitos mínimos, se definieron 4 perfiles respecto a los conocimientos que deben conseguir para completar de forma satisfactoria el proceso de jubilación:
\begin{itemize}
    \item Tiene derecho a SCOMP y tiene Beneficiarios Legales.
    \item Tiene derecho a SCOMP y no tiene Beneficiarios Legales.  
    \item No tiene derecho a SCOMP y tiene Beneficiarios Legales. 
    \item No tiene derecho a SCOMP y no tiene Beneficiarios Legales. 
\end{itemize}

Además, se consideran otras 2 variables que corresponden a su capacidad de aprendizaje y a su conocimiento previo sobre el sistema de pensiones, generando así 4 nuevos perfiles: 
\begin{itemize}
    \item Alto nivel de comprensión y aprendizaje y conocimiento previo del sistema de Pensiones. 
    \item Bajo nivel de comprensión y aprendizaje y conocimiento previo del sistema de Pensiones. 
    \item Alto nivel de comprensión y aprendizaje y bajo conocimiento previo del sistema de Pensiones. 
    \item Bajo nivel de comprensión y aprendizaje y bajo conocimiento previo del sistema de Pensiones. 
\end{itemize}

La tabla que resume esta información, junto con el tipo de contenido y secuencia que deberían ser considerados para cada perfil, se detalla a continuación. 




\begin{table}[h!]
\centering
\begin{tabular}{>{\bfseries}p{1.5 cm} p{4.5cm} p{4.5cm} p{4.5cm}}
\toprule
 & \textbf{Comprensión y Experiencia} & \textbf{Comprensión, s/experiencia} & \textbf{s/comprensión y experiencia} \\
 & Recursos Completos, sin restricción de complejidad & Recursos más claros, sin restricción de complejidad & Recursos más claros, con restricción de complejidad \\
\midrule
\textbf{Con SCOMP y BL} & 
\makecell[l]{Proceso de Pensión\\SCOMP\\Modalidades de pensión\\Beneficiarios\\Herencia\\Otros} &
\makecell[l]{Proceso de Pensión\\SCOMP\\Modalidades de pensión\\Beneficiarios\\Herencia\\Otros} &
\makecell[l]{Proceso de Pensión\\SCOMP\\Modalidades de pensión\\Beneficiarios\\Herencia\\Otros} \\
\midrule
\textbf{Con SCOMP y s/BL}  &  \makecell[l]{Proceso de Pensión\\SCOMP\\Modalidades de pensión\\Herencia\\Otros} &
\makecell[l]{Proceso de Pensión\\SCOMP\\Modalidades de pensión\\Herencia\\Otros} &
\makecell[l]{Proceso de Pensión\\SCOMP\\Modalidades de pensión\\Herencia\\Otros} \\
\midrule
\textbf{Sin SCOMP y BL}    &  \makecell[l]{Proceso de Pensión\\Retiros Programados\\Beneficiarios} &
\makecell[l]{Proceso de Pensión\\Retiros Programados\\Beneficiarios}  &
\makecell[l]{Proceso de Pensión\\Retiros Programados\\Beneficiarios}  
\\
\midrule
\textbf{Sin SCOMP y s/BL}  & \makecell[l]{Proceso de Pensión\\Retiros Programados} &
\makecell[l]{Proceso de Pensión\\Retiros Programados}  &
\makecell[l]{Proceso de Pensión\\Retiros Programados}  \\

\bottomrule
\end{tabular}
\caption{Categorización de perfiles de usuarios y tipo de recursos sugeridos}
\end{table}

