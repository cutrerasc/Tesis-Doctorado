\section{Idea experimental}

Dado que la atención de las personas es limitada y disminuye con el tiempo debido a la necesidad de priorizar a qué actividades destinarla, el objetivo principal de este trabajo es mantener la curva de atención lo más alta posible durante la navegación web para aprender sobre un proceso específico, en este caso, el proceso de jubilación. Para lograrlo, se busca evocar emociones durante la experiencia de navegación, con el fin de mejorar la atención y el aprendizaje. La siguiente tabla \ref{tab: Tabla de variables} muestra las posibles variables del problema. Mientras que en la tabla que sigue (tabla \ref{tabla}), se muestran los posibles tratamientos. 

\subsection{Variables del problema}
\begin{longtable}{| m{3cm} | m{3cm} | m{5cm} | m{3cm} |}
\caption{Posibles Variables}\label{Tabla de Variables}\\ 
\hline
\textbf{Nombre de Variable} & \textbf{Tipo de Variable} & \textbf{Explicación} & \textbf{Cómo se mide} \\ \hline
\endfirsthead
\hline
Nivel de Aprendizaje & Dependiente & Grado de conocimiento adquirido por el usuario después de usar la plataforma & Test de conocimientos Post - Test de conocimientos previos \\ \hline
Conocimiento Previo & Independiente & Grado de conocimiento del usuario antes de usar la plataforma & Test de conocimiento sobre el proceso de pensiones \\ \hline
Conocimiento Post & Independiente & Grado de conocimiento del usuario después de usar la plataforma & Test de conocimiento sobre el proceso de pensiones \\ \hline
Nivel de Motivación Intrínseca & Independiente & Grado de motivación interna para realizar una actividad sin recompensas externas & Ej: Intrinsic Motivation Inventory  \\ \hline
Memoria de Trabajo & Independiente & La habilidad del usuario para retener y manipular información mientras navega por la página. & Ej: "N-back test". \\ \hline
Emoción Previa & Independiente & Emoción del participante previa a usar la plataforma & Ej: PANAS \\ \hline
Emoción Post & Independiente & Emoción del participante después de usar la plataforma & Ej: PANAS \\ \hline 
Emoción evocada & Independiente & Tipo de emoción que se busca evocar en el usuario al iniciar la interacción con la plataforma (tratamiento) & EJ: Video \\ \hline

Edad & Control & La cantidad de años que tiene una persona desde su nacimiento & En años \\ \hline
Género & Control & La identificación de una persona en términos de su género & Masculino, Femenino, Otro \\ \hline

Habilidad Tecnológica & Independiente & La familiaridad del usuario con el uso de tecnología y la navegación web. & Cuestionarios sobre familiaridad y frecuencia de uso de la tecnología. \\ \hline

\end{longtable}

\subsection{Tratamientos}
Para cada una de las preguntas de investigación se plantean posibles experimentos. 
\begin{longtable}{| m{5cm} | m{2cm} | m{3cm} | m{5cm} |}
\caption{Posibles Tratamientos por Pregunta de Investigación } \label{tab: tratamiento}\\
\hline
\textbf{Pregunta de Investigación} & \textbf{Tipo de diseño} & \textbf{Variable dependiente} &\textbf{Condiciones experimentales} \\ \hline
\endfirsthead
\hline
\\¿Qué tipo de emociones incrementan de manera más significativa la atención en adultos mayores durante una secuencia de aprendizaje óptima sobre el proceso de pensiones? & Medidas repetidas (intrasujetos) & Aprendizaje & Evocación de emociones \begin{itemize}
    \item Confianza
    \item Interés
    \item Auto valencia
    \item Miedo
    \item Control (Sin evocación de emoción) 
\end{itemize} \\ \hline
¿De qué manera una secuencia de aprendizaje óptima que integra la evocación de emociones mejora la comprensión y retención de información sobre el proceso de pensiones en adultos mayores, en comparación a una secuencia estándar en términos de resultados de aprendizaje? & Grupos independientes (intersujetos) & \begin{itemize}
    \item Comprensión
    \item Retención de información 
\end{itemize} & Tipo de secuencia de aprendizaje: \begin{itemize}
    \item Secuencia Emocionalmente Optimizada
    \item Secuencia Estándar 
\end{itemize} \\ \hline
¿En qué medida la intensidad de la emoción evocada correlaciona con el nivel de atención sostenido por los adultos mayores a lo largo de la secuencia de aprendizaje? & Medidas repetidas (intrasujeto) & Nivel de atención sostenido durante la secuencia de aprendizaje & Diferentes niveles de intensidad emocional durante la secuencia de aprendizaje\begin{itemize}
    \item Baja Intensidad Emocional
    \item Media Intensidad Emocional
    \item Alta Intensidad Emocional
    \item Condición de Control
\end{itemize} \\ \hline
¿Cómo varía la atención en los adultos mayores al aprender sobre pensiones en función del momento en que se evocan las emociones durante la secuencia? &Medidas independientes (Intersujetos) & Nivel de atención sostenido durante la secuencia de aprendizaje. & Evocación de emociones en diferentes momentos de la secuencia de aprendizaje: \begin{itemize}
    \item Inicio de la secuencia
    \item Mitad de la secuencia
    \item Final de la secuencia
    \item Sin evocación
\end{itemize}  \\ \hline
\end{longtable}