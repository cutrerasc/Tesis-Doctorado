\newpage

\section*{Metodología Estudio 2}

%\textbf{Objetivo:} Comprender qué tipo de emociones (valencia positiva o negativa) dada una secuencia de aprendizaje optimizada para el usuario son mejores para incentivar y aumentar el aprendizaje en adultos en temática de pensiones. 
 \textbf{Objetivo:} Validar empíricamente la efectividad de la secuencia óptima de contenidos de aprendizaje y la incorporación de variables emocionales  sobre el proceso de jubilación para adultos.
 \begin{itemize}
     \item \textbf{Hipotesis 1:} Los adultos que aprenden con una secuencia óptima de contenidos muestran un aprendizaje mayor del proceso de jubilación en comparación con aquellos que utilizan una secuencia aleatoria.
     \item \textbf{Hipótesis 2:} La incorporación de variables emocionales en la secuencia óptima de contenidos aumenta el aprendizaje de los adultos, en comparación con secuencias que no incluyen estas variables.
 \end{itemize}

\textbf{Propuesta:}
Estudio experimental con diseño factorial 2x2 entre sujetos. Las dos variables independientes son:
\begin{itemize}
    \item  Tipo de secuencia: Secuencia de aprendizaje aleatoria vs. secuencia personalizada óptima. Cabe señalar que la secuencia optimizada será para cada perfil de usuario definida previamente de acuerdo a los resultados del modelo teórico, por lo que en este estudio no se pretende evaluar si la optimización realizada por el modelo es la idónea para el usuario. 
    \item  Emoción inducida: Esperanza (valencia positiva) vs. Miedo (valencia negativa).
\end{itemize}
   

   




Decisiones clave: 
\begin{itemize}
    \item La secuencia óptima teórica será diferente para cada perfil de usuario. ¿Utilizamos la secuencia optima para ese usuario? ¿Utilizamos una secuencia aleatoria cualquiera? ¿Probamos diferentes secuencias óptimas de forma aleatoria o basadas en las características del usuario?
    \item Las emociones a evocar serán de valencia positiva y negativa, pero ¿Cuáles usar? (Esperanza y miedo) 
    \item Cómo evocamos esas emociones, la literatura habla de ``Diseño emocional" (diseños del contenido que generen emociones diferentes, generalmente positivas y neutras). ``Feedback emocional" (Feedback del instructor con mensajes motivadores para el aprendiz), ``Motivaciones externas" (Videos, escritura propia que evoque alguna emoción específica, resultados diversos)  
\end{itemize}

\textbf{Propuesta}: 
Diseño experimental 2x3 con una única secuencia óptima de aprendizaje (discutible, dado que existirá una para cada perfil) vs aleatoria y emociones Valencia Positiva, Neutra y Negativa. 

Se mide el impacto de la emoción en el aprendizaje de la persona y el impacto de la secuencia optimizada. 