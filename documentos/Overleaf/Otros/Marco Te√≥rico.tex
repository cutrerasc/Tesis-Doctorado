\section{Introducción}

El proceso de jubilación es un tema complejo y de gran relevancia para los adultos mayores, especialmente en contextos donde el sistema de pensiones puede ser difícil de comprender y gestionar como es el caso de Chile. A medida que la población envejece, se vuelve crucial que los adultos mayores tengan acceso a información clara y comprensible sobre sus opciones de jubilación, permitiéndoles tomar decisiones informadas y autónomas. Sin embargo, la complejidad de los procesos administrativos y la barrera tecnológica que enfrentan muchos adultos mayores al interactuar con plataformas digitales son desafíos significativos que dificultan este aprendizaje.

En este contexto, la optimización de las secuencias de aprendizaje en plataformas digitales, especialmente aquellas dirigidas a adultos mayores, se presenta como una necesidad imperante. Diversos estudios han demostrado que la atención es un factor crítico en el proceso de aprendizaje, y que esta puede ser influenciada por el estado emocional del individuo. Así, la evocación de emociones específicas durante la interacción con una plataforma de aprendizaje podría ser una estrategia efectiva para mejorar tanto la atención como la retención de información en los adultos mayores.

Este proyecto de investigación se centra en explorar cómo la integración de emociones en el diseño de una secuencia de aprendizaje puede aumentar la atención de los usuarios mejorando su comprensión y retención, específicamente en el proceso de pensiones. Al combinar principios de diseño centrado en el usuario, teorías del aprendizaje, y estudios sobre el impacto de las emociones en la cognición, este estudio busca desarrollar un modelo de secuenciación optimizado que no solo sea accesible y efectivo, sino que también esté alineado con las necesidades cognitivas y emocionales de los adultos mayores.

Así, este trabajo propone investigar el impacto de la evocación de emociones en la secuencia de aprendizaje sobre el proceso de jubilación en adultos mayores, con el objetivo final de mejorar su experiencia de usuario y facilitar un aprendizaje más efectivo y duradero.

\subsection*{Preguntas de investigación y objetivos}
A continuación se proponen 4 posibles preguntas de investigación con sus respectivos objetivos generales y específicos. 

\begin{enumerate}
    \item ¿Qué tipo de emociones incrementan de manera más significativa la atención en adultos mayores durante una secuencia de aprendizaje óptima sobre el proceso de pensiones?
    \begin{itemize}
        \item \textbf{Objetivo General:} Identificar los tipos de emociones que incrementan significativamente la atención en adultos mayores durante una secuencia de aprendizaje sobre el proceso de pensiones.
        \item \textbf{Objetivos específicos:}
        \begin{itemize}
            \item Determinar las emociones más comunes que se pueden evocar en adultos mayores en el contexto del aprendizaje digital.
            \item Medir el nivel de atención de los adultos mayores en relación con diferentes emociones evocadas durante la secuencia de aprendizaje.
            \item Comparar la efectividad de diferentes emociones en el aumento de la atención durante la interacción con la plataforma.
        \end{itemize}
    \end{itemize}
    \item ¿De qué manera una secuencia de aprendizaje óptima que integra la evocación de emociones mejora la comprensión y retención de información sobre el proceso de pensiones en adultos mayores, en comparación a una secuencia estándar en términos de resultados de aprendizaje?
    \begin{itemize}
        \item \textbf{Objetivo general:} Evaluar y comparar cómo la integración de emociones en una secuencia de aprendizaje óptima mejora la comprensión y retención de la información sobre el proceso de pensiones en adultos mayores.
        \item \textbf{Objetivos específicos:}
        \begin{itemize}
            \item Diseñar una secuencia de aprendizaje optimizada que integre la evocación de emociones.
            \item Analizar la comprensión inicial de la información sobre pensiones.
            \item Medir y comparar los resultados de aprendizaje y la retención de información entre los grupos que utilizan la secuencia optimizada y la secuencia estándar.
            \item Analizar las diferencias en la retención y comprensión de la información entre ambos grupos.
        \end{itemize}
    \end{itemize}
    \item ¿En qué medida la intensidad de la emoción evocada correlaciona con el nivel de atención sostenido por los adultos mayores a lo largo de la secuencia de aprendizaje?
    \begin{itemize}
        \item \textbf{Objetivo general:} Determinar la correlación entre la intensidad de la emoción evocada y el nivel de atención sostenido por los adultos mayores durante una secuencia de aprendizaje.
        \item \textbf{Objetivos específicos:}
        \begin{itemize}
            \item Identificar las métricas adecuadas para medir la intensidad de las emociones y el nivel de atención en adultos mayores.
            \item Evaluar la relación entre los diferentes niveles de intensidad emocional y la atención sostenida a lo largo de la secuencia de aprendizaje.
            \item Comparar la atención sostenida en situaciones de alta y baja intensidad emocional.
        \end{itemize}    
    \end{itemize}
    \item ¿Cómo varía la atención en los adultos mayores al aprender sobre pensiones en función del momento en que se evocan las emociones durante la secuencia?
    \begin{itemize}
        \item \textbf{Objetivo General:} Analizar cómo varía la atención de los adultos mayores en función del momento en que se evocan las emociones durante una secuencia de aprendizaje sobre el proceso de pensiones.
        \item \textbf{Objetivos específicos:}
        \begin{itemize}
            \item Diseñar experimentos que evoquen emociones en diferentes momentos de la secuencia de aprendizaje.
            \item Medir la atención de los adultos mayores en cada momento de la secuencia donde se evocan emociones.
            \item Determinar el momento más efectivo para la evocación de emociones que maximice la atención durante la secuencia de aprendizaje.
        \end{itemize}
    \end{itemize}
\end{enumerate}



\section{Fundamentos de la Interfaz de Usuario (UI), Experiencia de Usuario (UX) y Diseño Emocional}

La interfaz de usuario (UI) y la experiencia de usuario (UX) son conceptos esenciales en el diseño de sistemas interactivos que, aunque están interrelacionados, abordan diferentes aspectos de la interacción del usuario con el producto. La UI se enfoca en los elementos gráficos y de interacción, tales como botones, menús y otros componentes visuales que permiten la interacción directa del usuario con el sistema \cite{ramadhanti_use_2023}. En contraste, la UX se refiere a la percepción global del usuario al interactuar con el producto, incluyendo aspectos emocionales, funcionales y de satisfacción general. La UX considera no solo la eficiencia y la facilidad de uso, sino también cómo el producto satisface las necesidades y expectativas del usuario a lo largo del tiempo \cite{hassenzahl_user_2006, berni_definition_2021}. Sin embargo, a pesar de su importancia, no se ha logrado un consenso universal en la definición de UX, lo cual refleja la diversidad de enfoques y aplicaciones dentro del campo \cite{hassenzahl_user_2006,berni_definition_2021}. Una buena UI/UX puede transformar la interacción del usuario en una experiencia fluida y agradable, logrando un balance entre funcionalidad y satisfacción emocional \cite{ramadhanti_use_2023, hassenzahl_user_2006, berni_definition_2021}.

Por otro lado, la historia y evolución de las interfaces de usuario (UI) ha transformado significativamente la interacción humano-computadora a lo largo de las décadas. Desde las primeras investigaciones en los años 60, como el sistema Sketchpad de Ivan Sutherland, que permitía la manipulación directa de gráficos \cite{myers_brief_1998}, hasta los modernos sistemas de interacción táctil y de voz, las UI han evolucionado para volverse más intuitivas y accesibles. Inicialmente, los esfuerzos se centraron en aplicaciones críticas como las cabinas de aviones y las salas de control industrial, donde la precisión y la eficiencia eran esenciales \cite{shneiderman_paradigm_2017}. Con el tiempo, la llegada de las interfaces gráficas de usuario (GUI) en los años 80, popularizadas por sistemas como Xerox Star y Apple Macintosh, democratizó el acceso a la tecnología, haciendo que las computadoras fueran más amigables y usables para el público general \cite{gould_how_1988}. La evolución continua de las UI ha sido impulsada por una combinación de avances tecnológicos, como el aumento de la velocidad y la reducción de costos de los chips, y un enfoque creciente en la usabilidad y la experiencia del usuario (UX), fundamentado en principios de diseño centrados en el usuario y en pruebas iterativas continuas \cite{jorgensen_taking_2008}. Esta trayectoria ha permitido la creación de sistemas que no solo son funcionales, sino también agradables y eficientes.

Los principios básicos de diseño de UI y UX son fundamentales para la creación de interfaces de usuario que sean tanto eficientes como satisfactorias. Según Nielsen y Molich \citeyear{nielsen_heuristic_1990}, estos principios incluyen el diálogo simple y natural, el uso del lenguaje del usuario, la minimización de la carga de memoria del usuario, la consistencia, la retroalimentación, la provisión de salidas claramente marcadas, los atajos, los buenos mensajes de error y la prevención de errores. La evaluación heurística, que aplica estos principios, ha demostrado ser particularmente eficaz cuando se combinan las evaluaciones de múltiples evaluadores, permitiendo identificar una mayor cantidad de problemas de usabilidad.

Por otro lado, Norman \citeyear{norman_design_1983} destaca la importancia de principios generales que deben ser útiles y aplicables a una amplia gama de tecnologías, utilizando métodos cuantitativos para evaluar las relaciones de compensación entre diferentes atributos del diseño. Además, Gould \citeyear{gould_how_1988} enfatiza la necesidad de enfocarse temprana y continuamente en los usuarios, integrando todos los aspectos de la usabilidad desde el principio y realizando pruebas de usuario iterativas para mejorar el diseño.

Ruiz, Serral y Snoeck \citeyear{ruiz_unifying_2021} abordan la unificación de los principios de diseño funcional de UI, identificando 36 principios clave que proporcionan una guía clara para diseñadores y educadores. Estos principios incluyen la visibilidad, la consistencia, la prevención de errores y la carga cognitiva mínima. Hamidli \citeyear{hamidli_introduction_2023} agrega que un buen diseño UI/UX debe ser accesible, consistente y debe proporcionar retroalimentación clara y oportuna.

Además, Stadler \citeyear{stadler_how_2022} explora cómo los principios psicológicos pueden influir en el diseño de UI y UX, utilizando técnicas como el reconocimiento de patrones y la gamificación para mejorar la experiencia del usuario. Finalmente, Vergara y Poma \citeyear{heidy_elizabeth_vergara_zurita_ux_2023} enfatizan la aplicación de metodologías UX/UI para mejorar la interacción y usabilidad de sitios web de comercio electrónico, demostrando que la implementación adecuada de estos principios contribuye significativamente a la competitividad y la satisfacción del usuario.

Los tipos de interfaces de usuario (UI) han evolucionado significativamente, abarcando diversas formas y tecnologías que mejoran la interacción entre los usuarios y los sistemas. Entre los tipos más comunes se encuentran las interfaces gráficas de usuario (GUI), que utilizan elementos visuales como ventanas, iconos y menús para facilitar la navegación y la ejecución de tareas, siendo esenciales en la mayoría de las aplicaciones modernas por su facilidad de uso y eficiencia \cite{yen_user_1998}. Otro tipo destacado son las interfaces de usuario inteligentes (IUI), que integran inteligencia artificial para adaptar y personalizar la interacción según el contexto y las necesidades del usuario, incrementando así la efectividad y eficiencia del sistema \cite{goncalves_systematic_2019}. Además, las interfaces basadas en presentaciones utilizan objetos pictóricos o textuales estructurados para transmitir información de manera efectiva, modelando la interfaz como un medio de comunicación compartido \cite{ciccarelli_presentation_1981}. También es relevante considerar las interfaces de usuario implantadas, que permiten interacciones siempre disponibles mediante dispositivos implantados bajo la piel, aunque presentan desafíos técnicos y médicos significativos \cite{holz_implanted_2012}. Finalmente, el diseño centrado en el usuario, que considera la usabilidad, accesibilidad y consistencia, es crucial para crear interfaces intuitivas y efectivas que satisfagan las necesidades de los usuarios y los objetivos del sistema \cite{cherry_designing_1992}.

\subsection{Diseño Emocional}

\section{Teorías sobre Emociones en la Interacción con la Tecnología}

Las teorías clásicas y modernas sobre las emociones reflejan una evolución considerable en la comprensión de estos fenómenos complejos. Las teorías clásicas, como las propuestas por William James y Carl Lange, postulaban que las emociones eran respuestas fisiológicas automáticas a estímulos externos, sin considerar significativamente el papel de la cognición \cite{james_what_1948, lange_emotions_1922}. En contraste, las teorías modernas de evaluación cognitiva, desarrolladas por investigadores como Richard Lazarus, argumentan que las emociones resultan de procesos de evaluación cognitiva, en los cuales los individuos valoran la relevancia y el significado de los eventos en relación con sus metas personales y bienestar \cite{lazarus_emotion_1991}. Este enfoque cognitivo destaca que las emociones son procesos complejos que involucran evaluaciones cognitivas, motivacionales y relacionales \cite{ellsworth_implications_1991}. Por ejemplo, el trabajo de Joseph LeDoux en "The Emotional Brain" subraya cómo las emociones como el miedo son procesadas a través de sistemas neurales específicos como la amígdala, integrando aspectos fisiológicos y cognitivos \cite{ledoux_emotional_1998}. Además, el enfoque evolutivo de la emoción propuesto por Paul Ekman identifica emociones básicas universales expresadas facialmente, sugiriendo una base biológica compartida para las respuestas emocionales \cite{ekman_emotion_1972}. En conjunto, estas teorías modernas proporcionan un marco comprensivo que integra la fisiología, la cognición y la evolución para entender mejor cómo se generan y experimentan las emociones.

La relación entre emociones y tecnología es un campo de estudio multifacético que examina cómo las tecnologías digitales afectan y son afectadas por las emociones humanas. Investigaciones han demostrado que las emociones juegan un papel crucial en la adopción y el uso de la tecnología. Por ejemplo, Beaudry y Pinsonneault \citeyear{beaudry_other_2010} encontraron que las emociones como la felicidad y la ansiedad tienen efectos directos e indirectos significativos en el uso de tecnologías de la información, mediadas por comportamientos de adaptación. Además, el estudio de Jarrell et al. \citeyear{jarrell_success_2017} subraya la importancia de las emociones retrospectivas en el rendimiento académico en entornos de aprendizaje digital, donde las emociones positivas como el disfrute y el orgullo están asociadas con un mejor desempeño.

Las tecnologías diseñadas para mejorar el estado de ánimo, como las aplicaciones de mindfulness y los videojuegos, también demuestran el impacto directo de la tecnología en la regulación emocional \cite{wadley_mood-enhancing_2016}. Por otro lado, Krueger y Osler \citeyear{krueger_engineering_2019} argumentan que el Internet y los dispositivos conectados actúan como nichos tecno-sociales que no solo soportan funciones cognitivas sino que también regulan nuestras emociones, introduciendo tanto beneficios como desafíos. Este campo también se extiende al uso de redes sociales y su capacidad para reconocer y manipular emociones, planteando preocupaciones sobre privacidad y seguridad emocional \cite{andalibi_human_2020}. La comprensión de estas dinámicas es crucial para diseñar tecnologías que no solo sean funcionales, sino también emocionalmente inteligentes y éticamente responsables.

Las emociones son un componente esencial en la interacción humano-computadora (HCI) y la experiencia del usuario (UX), influyendo significativamente en la forma en que los usuarios perciben y utilizan la tecnología. Investigaciones recientes han desarrollado y validado diversos modelos y herramientas para medir las emociones de los usuarios durante estas interacciones. Por ejemplo, el reconocimiento de emociones multimodal ha demostrado una alta precisión en la identificación de estados emocionales específicos, mejorando la evaluación de UX \cite{razzaq_hybrid_2023}. En la misma línea, el Cuestionario de Emociones Discretas (DEQ) se ha establecido como una herramienta válida y fiable para medir emociones discretas en contextos tecnológicos \cite{harmon-jones_discrete_2016}.

El enfoque basado en modelos mentales relacionados con la interacción (irMM) permite evaluar las emociones y expectativas de los usuarios, proporcionando una comprensión más profunda de sus experiencias \cite{ahram_user_2018}. Además, la investigación ha destacado la importancia de considerar las emociones en contextos colaborativos y culturales, ya que estas pueden ser co-construidas y experimentadas de manera diversa en entornos grupales \cite{shami_measuring_2008}. Este enfoque es fundamental para diseñar sistemas que no solo sean funcionales, sino que también proporcionen una experiencia emocionalmente satisfactoria \cite{lottridge_designing_2009}.

Por otro lado, el análisis de textos en redes sociales mediante aprendizaje automático ha permitido medir con precisión las expresiones emocionales, ofreciendo una nueva perspectiva para la evaluación de UX \cite{brady_theory-driven_2021}. A través de una revisión sistemática de las prácticas de medición en la investigación de UX, se han identificado prácticas de medición cuestionables y se han proporcionado recomendaciones para mejorar la adherencia a las mejores prácticas en el uso de escalas \cite{perrig_measurement_2024}. Estas técnicas avanzadas permiten a los investigadores y diseñadores capturar la complejidad de las emociones humanas y mejorar el diseño de interfaces tecnológicas que respondan mejor a las necesidades emocionales de los usuarios.

\section{Impacto de las Emociones en el Aprendizaje}

Las teorías del aprendizaje han evolucionado significativamente, destacando la importancia de la interacción social, el contexto cultural y las experiencias personales en el desarrollo cognitivo. Vygotsky \citeyear{vygotskij_mind_1981} subraya que las funciones psicológicas superiores se desarrollan a través de la mediación de herramientas culturales y la interacción social, conceptualizando la Zona de Desarrollo Próximo (ZDP) como el espacio donde el aprendizaje es optimizado mediante la guía de individuos más capaces. Bandura \citeyear{bandura_social_1977} complementa esta perspectiva con su teoría del aprendizaje social, que enfatiza la observación y la imitación de comportamientos dentro de un contexto social como procesos fundamentales para la adquisición de nuevas conductas. Por otro lado, Kolb \citeyear{kolb_experiential_2015} propone un modelo de aprendizaje experiencial que describe el aprendizaje como un ciclo continuo de experiencias concretas, observación reflexiva, conceptualización abstracta y experimentación activa. Este enfoque se ve reflejado también en la teoría de Lave y Wenger \citeyear{lave_situated_1991} sobre el aprendizaje situado, que argumenta que el aprendizaje ocurre a través de la participación activa en comunidades de práctica. Gagné \citeyear{gagne_conditions_1970}, por su parte, identifica diferentes tipos de aprendizaje y las condiciones necesarias para cada uno, enfatizando la importancia de un diseño instruccional que tenga en cuenta tanto las capacidades previas del aprendiz como el entorno educativo. Estas teorías, integradas, proporcionan un marco comprensivo que reconoce la complejidad del proceso de aprendizaje y la necesidad de enfoques educativos que consideren los aspectos cognitivos, emocionales, y sociales del individuo \cite{bransford_how_2000, illeris_towards_2003}.

La relación entre emociones y procesos cognitivos es compleja y multifacética, evidenciando una profunda interconexión que influye significativamente en diversos aspectos de la cognición humana. Las emociones modulan la percepción y la atención, facilitando la codificación y recuperación de información, y pueden tanto mejorar como perjudicar el aprendizaje y la retención a largo plazo \cite{tyng_influences_2017}. Estudios neurocientíficos han demostrado que la amígdala, la corteza prefrontal y el hipocampo interactúan de manera integrada para procesar y almacenar memorias emocionales, asegurando que las experiencias emocionales se recuerden con mayor claridad y precisión \cite{pessoa_impact_2013}. Además, las emociones positivas tienden a mejorar la memoria y la atención, mientras que las emociones negativas pueden interferir en estos procesos, aunque también pueden aumentar el enfoque en el material de estudio \cite{imbir_heart_2016}. Este impacto se observa en cómo los estados emocionales afectan la memoria de trabajo y las funciones ejecutivas, influyendo en la toma de decisiones y la resolución de problemas \cite{inzlicht_emotional_2015}. La integración de teorías de la emoción con modelos computacionales de ciencia cognitiva proporciona una comprensión más precisa y cuantitativa de cómo las emociones influyen en la cognición y el comportamiento, lo que es esencial para el diseño de intervenciones educativas efectivas que promuevan un aprendizaje óptimo \cite{perlovsky_higher_2020}.

Los estudios empíricos han demostrado que las emociones juegan un papel crucial en el proceso de aprendizaje, influenciando tanto el rendimiento académico como el desarrollo personal de los estudiantes. Emociones positivas como el disfrute y la sorpresa tienden a mejorar la motivación, la memoria y el compromiso con las tareas académicas \cite{yin_effects_2023,linnenbrink_role_2007}. Por otro lado, emociones negativas como la ansiedad y la frustración pueden consumir recursos cognitivos y reducir la eficiencia del aprendizaje, aunque en algunos casos pueden también motivar a los estudiantes a superar desafíos \cite{rowe_understanding_2018, ge_emotion_2021}. La inteligencia emocional (EI) ha sido identificada como un factor mediador significativo que influye en los resultados de aprendizaje, facilitando la autorregulación emocional y promoviendo un ambiente de aprendizaje positivo \cite{shafait_assessment_2021}. La implementación de estrategias que regulen las emociones y promuevan un ambiente de apoyo emocional puede potenciar el rendimiento académico y la satisfacción de los estudiantes con su experiencia educativa \cite{tan_influence_2021,mazars_influence_2023}. En conjunto, estos hallazgos subrayan la importancia de considerar las emociones en el diseño de entornos educativos y en la formación de los docentes para maximizar el impacto positivo en el aprendizaje y el desarrollo académico de los estudiantes.

\section{Interacción Humano-Computadora (HCI)}

La historia y evolución de la Interacción Humano-Computadora (HCI) ha sido un viaje significativo desde sus inicios en las décadas de 1950 y 1960, con innovaciones clave como el Sketchpad de Ivan Sutherland, que introdujo la manipulación directa de objetos gráficos \cite{myers_brief_1998}, y el desarrollo del ratón por Douglas Engelbart en el Stanford Research Laboratory \cite{myers_brief_1998}. Durante las décadas de 1970 y 1980, HCI se consolidó a través de la revolución cognitiva y la ingeniería de factores humanos, destacándose en lugares como Xerox PARC, donde se desarrollaron interfaces gráficas de usuario revolucionarias como el Xerox Star y el Apple Macintosh \cite{grudin_moving_2012,myers_brief_1998}. La evolución del campo también incluyó avances en la edición de texto y la tecnología de hipertexto, fundamentales para el desarrollo de la World Wide Web \cite{myers_brief_1998}. En años recientes, el enfoque en el diseño centrado en el usuario y las pruebas de usabilidad ha sido crucial para desarrollar tecnologías intuitivas y accesibles \cite{jain_human-computer_2023}. Actualmente, la investigación en HCI abarca tecnologías emergentes como la realidad virtual, la inteligencia artificial y las interfaces adaptativas, subrayando la importancia continua de la colaboración entre universidades, laboratorios corporativos y el apoyo gubernamental para el desarrollo de interfaces avanzadas \cite{harrison_three_2007,myers_brief_1998}. Esta evolución refleja un compromiso constante con mejorar la interacción entre humanos y computadoras, abordando tanto los aspectos técnicos como los sociales y organizacionales del diseño de tecnología.


\section{Interacción con Interfaces de Usuario en Adultos Mayores}
\subsection{Características y Necesidades de los Adultos Mayores}

Los adultos mayores experimentan una serie de cambios cognitivos y físicos asociados con el envejecimiento que afectan su calidad de vida y capacidad para realizar actividades cotidianas. Cognitivamente, se observa una disminución en la velocidad de procesamiento, memoria episódica y función ejecutiva \cite{hughes_change_2018, murman_impact_2015, van_patten_appreciating_2021, american_psychological_association_older_2021}. Aunque la memoria semántica y las habilidades cristalizadas, que incluyen el conocimiento acumulado y las habilidades adquiridas, se mantienen relativamente estables, las habilidades fluidas, que requieren procesamiento cognitivo en tiempo real, muestran un declive más significativo \cite{harada_normal_2013}. Físicamente, los adultos mayores enfrentan una reducción en la fuerza muscular, movilidad y capacidad cardiorrespiratoria, así como una pérdida de volumen cerebral y cambios estructurales en la materia gris y blanca \cite{song_physical_2023, preston_physiology_2021, hughes_change_2018}. La disminución en la aptitud física y el equilibrio también afecta significativamente su independencia y calidad de vida \cite{wrights_assessing_2015}. A pesar de estos desafíos, la participación en actividades físicas y sociales puede mitigar algunos efectos del envejecimiento, destacando la importancia de intervenciones que promuevan la salud física y cognitiva para mejorar la calidad de vida en esta población \cite{martins_observational_2024, wrights_assessing_2015}.

Los adultos mayores presentan características y necesidades específicas en relación con la tecnología, siendo la tecnofobia y la experiencia previa con la tecnología factores determinantes en su adopción y uso. La tecnofobia, definida como el miedo o la ansiedad hacia las tecnologías, afecta significativamente a este grupo, limitando su disposición a utilizar herramientas digitales que podrían mejorar su calidad de vida \cite{hogan_technophobia_2006}. Las barreras incluyen la falta de habilidades cognitivas, la educación previa y la complejidad percibida de los dispositivos tecnológicos \cite{longe_technophobia_2007, thalib_older_2019}. Sin embargo, la influencia social y las percepciones positivas del envejecimiento pueden moderar estos efectos, promoviendo una mayor adopción tecnológica \cite{xi_when_2022}. Los programas educativos y de capacitación, como el programa HELPS-seniors, han demostrado ser efectivos para reducir la tecnofobia y mejorar la alfabetización digital entre los adultos mayores con baja educación \cite{wang_overcoming_2015}. Además, la proximidad física intergeneracional puede desempeñar un papel crucial, ya que una mayor distancia física durante el aprendizaje puede disminuir la percepción de amenaza y reducir la tecnofobia \cite{xi_when_2022}. En resumen, abordar la tecnofobia y mejorar la experiencia previa con la tecnología a través de intervenciones educativas y apoyo social es esencial para fomentar la inclusión digital de los adultos mayores.

\subsection{Diseño de Interfaces para Adultos Mayores}

El diseño de interfaces para adultos mayores se fundamenta en principios de accesibilidad e inclusión que consideran las limitaciones físicas, sensoriales y cognitivas de esta población. La simplicidad y claridad son esenciales para reducir la carga cognitiva y facilitar la navegación, haciendo uso de fuentes grandes y de alto contraste, botones de fácil clic y retroalimentación clara \cite{wang_research_2020}. Además, el diseño debe incluir mensajes de ayuda contextuales y confirmatorios para aumentar la confianza del usuario y facilitar la comprensión del sistema \cite{zajicek_successful_2004}.

Es crucial adoptar un enfoque de Diseño Centrado en el Usuario que se adapte a la diversidad dinámica de habilidades entre los adultos mayores, lo que implica la creación de interfaces adaptativas que se ajusten a las necesidades cambiantes de los usuarios \cite{gregor_designing_2001}. La participación activa de los adultos mayores en el proceso de diseño, a través de metodologías participativas, asegura que las herramientas tecnológicas desarrolladas respondan verdaderamente a sus necesidades y preferencias, promoviendo así su autonomía y calidad de vida \cite{martin-hammond_engaging_2018}.

El uso de patrones de diseño permite encapsular buenas prácticas y proporcionar un marco accesible para diseñadores, facilitando la creación de interfaces intuitivas y usables \cite{zajicek_successful_2004}. En definitiva, un diseño inclusivo y accesible para adultos mayores no solo mejora su experiencia de usuario, sino que también fomenta su inclusión digital y apoyo en la vida diaria.

El diseño de interfaces para adultos mayores en el contexto de la interacción persona-ordenador requiere una consideración cuidadosa de las necesidades específicas de este grupo demográfico, cuyas capacidades físicas y cognitivas pueden variar significativamente debido al envejecimiento. La creciente dependencia de la tecnología para acceder a servicios básicos, como la programación de citas médicas y la obtención de información pública, destaca la importancia de desarrollar interfaces accesibles y usables que prevengan la exclusión social y promuevan la independencia de los adultos mayores \cite{zajicek_successful_2004}. Estudios han demostrado que los patrones de diseño que consideran la diversidad dinámica de las capacidades de los usuarios mayores no solo mejoran la usabilidad, sino que también benefician a todos los usuarios al hacer las interfaces más intuitivas y accesibles \cite{morris_user_1994, zajicek_successful_2004}. Además, se ha evidenciado que las interfaces adaptativas, que se ajustan a las necesidades cambiantes de los usuarios, pueden reducir significativamente el esfuerzo de interacción, mejorando la experiencia del usuario y aumentando la frecuencia de uso de las aplicaciones móviles entre los adultos mayores \cite{romero_long-term_2020}. Estas interfaces deben ser diseñadas con un enfoque inclusivo que permita a los usuarios mayores participar activamente en la interacción, utilizando ayudas como asistentes de voz para facilitar la navegación en entornos digitales complejos \cite{yu_where_2023}.

El diseño de interfaces para adultos mayores se ha convertido en un área de interés creciente, dada la importancia de la accesibilidad y la usabilidad en la interacción de este grupo demográfico con la tecnología. Los estudios han demostrado que los adultos mayores enfrentan desafíos únicos al interactuar con interfaces digitales, como dificultades cognitivas y físicas que pueden hacer que la experiencia de usuario sea complicada y frustrante \cite{chiu_redesigning_2019, neves_usability_2019, abdullah_usability_2018}. Para abordar estos desafíos, se han propuesto diversas estrategias de diseño, que incluyen la simplificación de la navegación, la adaptación de la interfaz a las limitaciones físicas y cognitivas, y el uso de pruebas de usabilidad específicas para este grupo \cite{correia_design_2014, salman_design_2023}. La investigación subraya la importancia de diseñar interfaces que no solo sean funcionales, sino que también satisfagan las necesidades emocionales y sociales de los adultos mayores, permitiéndoles una interacción más satisfactoria y menos estresante con la tecnología \cite{silva_experts_2020, wang_research_2020}. En resumen, la evaluación de la usabilidad en adultos mayores es crucial para garantizar que las interfaces diseñadas para ellos sean efectivas, accesibles y, sobre todo, inclusivas.
   
\subsection{Emociones y Experiencia de Usuario en Adultos Mayores}

El impacto emocional de la tecnología en los adultos mayores es un aspecto crucial de la experiencia de usuario (UX) que influye significativamente en la adopción y el uso de nuevas tecnologías. Las emociones negativas, como la ansiedad y la frustración, a menudo surgen debido a la complejidad y la falta de personalización en los dispositivos tecnológicos, lo que actúa como una barrera para su adopción en este grupo etario \cite{kim_emotional_2012, smith_digital_2021}. Sin embargo, se ha demostrado que cuando las tecnologías se diseñan teniendo en cuenta las necesidades emocionales de los usuarios mayores, pueden fomentar una mayor aceptación y mejorar la calidad de vida, como es el caso de los sistemas de alarma personal que integran un enfoque de diseño emocional \cite{pedell_dont_2015}. Además, las herramientas de evaluación de UX, como la Escala de Usabilidad del Sistema (SUS) y el Cuestionario de Experiencia de Usuario (UEQ), sirven para mejorar la experiencia emocional y la adaptación de los adultos mayores a las tecnologías de salud digital \cite{takano_user_2023}. En conjunto, estos hallazgos sugieren que el diseño tecnológico que considera tanto las dimensiones funcionales como emocionales es esencial para maximizar el impacto positivo de la tecnología en la vida de los adultos mayores.

Para mejorar la experiencia de usuario (UX) en adultos mayores y fomentar emociones positivas, es crucial adoptar un enfoque de diseño que considere tanto las limitaciones físicas y cognitivas como las necesidades emocionales de este grupo. Estudios recientes destacan la importancia de integrar el bienestar digital en el diseño de tecnologías, asegurando que sean accesibles, intuitivas y capaces de reducir la ansiedad que la digitalización puede generar en los mayores \cite{vilpponen_designing_2020}. Además, la participación de los adultos mayores en el proceso de diseño permite ajustar las tecnologías a sus necesidades específicas, lo que no solo mejora la usabilidad, sino que también refuerza el sentido de pertenencia y satisfacción emocional \cite{fischer_importance_2020}. El diseño que promueve el apego emocional y la creatividad en la interacción con la tecnología puede transformar la experiencia del usuario en una fuente de bienestar, motivando un uso más activo y positivo de las tecnologías \cite{lee_creativity_2019}. Asimismo, herramientas como dispositivos tangibles para el registro de emociones pueden apoyar la salud emocional de los adultos mayores, facilitando la reflexión y el autoconocimiento, lo que contribuye a un bienestar emocional sostenido \cite{gooch_designing_2022}. Estos enfoques sugieren que la UX debe ir más allá de la funcionalidad técnica, incorporando estrategias que fomenten un entorno emocionalmente positivo y un compromiso significativo con la tecnología.

Los estudios empíricos sobre emociones y experiencia de usuario en adultos mayores han demostrado que esta población prioriza experiencias emocionales positivas, lo que influye significativamente en su interacción con las tecnologías y en el proceso de aprendizaje. Se ha encontrado que el diseño de interfaces que evocan emociones positivas puede mejorar la retención y el aprendizaje, ya que las personas mayores tienden a recordar mejor las experiencias positivas y a evitar o no procesar completamente las negativas \cite{kremer_studying_2016}. Además, la inducción de emociones positivas a través del diseño emocional en entornos de aprendizaje multimedia facilita el procesamiento cognitivo y mejora la comprensión y la transferencia de conocimientos, lo que es relevante para su aplicación en adultos mayores \cite{um_emotional_2012}. Finalmente, la inmersión emocional en entornos educativos virtuales ha demostrado incrementar la motivación y el aprendizaje, destacando la importancia de diseñar experiencias que consideren tanto las emociones como las capacidades cognitivas y sensoriales de los adultos mayores \cite{lie_developing_2023}.


\subsection{Impacto en el Aprendizaje de Adultos Mayores}

El impacto de las teorías del aprendizaje en el contexto de la educación de adultos mayores es significativo, particularmente cuando se aplican estrategias pedagógicas adaptadas a sus necesidades específicas. Por ejemplo, la teoría del aprendizaje significativo de Ausubel y la teoría sociocultural de Vygotsky han demostrado ser eficaces en intervenciones educativas destinadas a mejorar el conocimiento sobre alimentación saludable entre adultos mayores, como lo demuestran Ortiz Segarra et al. \citeyear{ortiz_segarra_impact_2023}. Estas teorías promueven un aprendizaje en el que la nueva información se conecta con conceptos preexistentes, facilitando una mayor retención y comprensión del contenido, especialmente cuando se utiliza material culturalmente adaptado y estrategias interactivas. Además, la teoría OPTIMAL, que optimiza el rendimiento a través de la motivación intrínseca y la atención externa, también ha mostrado resultados positivos en el aprendizaje motor de los adultos mayores, destacando la importancia de considerar factores como las expectativas y el apoyo a la autonomía durante el proceso de enseñanza \cite{khalaji_optimizing_2024}. Estos enfoques teóricos subrayan la relevancia de personalizar las estrategias de enseñanza para maximizar la eficacia del aprendizaje en esta población, abordando tanto las dimensiones cognitivas como emocionales del aprendizaje.
   
El aprendizaje digital en adultos mayores es un proceso complejo influenciado por diversas barreras y facilitadores que pueden afectar significativamente su eficacia. Entre las principales barreras se encuentran la falta de confianza en las propias habilidades digitales, el desconocimiento de la tecnología, y las limitaciones físicas y cognitivas propias del envejecimiento, como la disminución de la destreza manual y la vista \cite{wilson-menzfeld_learning_2023, harris_older_2022}. Además, factores como la preferencia por el aprendizaje presencial y las preocupaciones sobre la privacidad y la seguridad de los datos también limitan la adopción de tecnologías digitales por parte de los adultos mayores \cite{Keohane_barriers_2022}. Sin embargo, existen importantes facilitadores que pueden mitigar estas barreras, tales como la personalización del aprendizaje, el apoyo continuo y la construcción de relaciones sólidas entre instructores y participantes \cite{wilson-menzfeld_learning_2023}. La personalización de la capacitación digital, adaptada a las necesidades individuales de los adultos mayores, ha demostrado ser especialmente eficaz para empoderar a esta población y fomentar la continuación del aprendizaje digital, lo que sugiere que las intervenciones deben diseñarse considerando estas especificidades para maximizar su impacto positivo \cite{wilson-menzfeld_learning_2023,gedvilaite-kordusiene_digital_2023}.

El impacto del aprendizaje en los adultos mayores se ve significativamente potenciado cuando se aplican estrategias de diseño que consideran sus particularidades cognitivas y motivacionales. Los estudios han demostrado que la implementación de recursos personalizados, como la inclusión de etiquetas de texto y la simplificación del contenido en módulos secuenciales, mejora la retención y comprensión en esta población \cite{pappas_cognitive-based_2019, song_developing_2024}. Además, la utilización de plataformas de aprendizaje en línea, como los MOOCs, adaptadas para resolver problemas específicos de los adultos mayores y fomentar la interacción social, ha mostrado ser efectiva para mantener la motivación y mejorar el bienestar cognitivo y emocional \cite{xiong_older_2019}. La gamificación y la enseñanza de múltiples habilidades simultáneamente también han sido identificadas como estrategias exitosas para mejorar tanto las capacidades cognitivas como la independencia funcional en adultos mayores \cite{leanos_impact_2023,kappen_design_2016}. Estas intervenciones no solo promueven un aprendizaje efectivo, sino que también contribuyen a la mejora de la calidad de vida y la autonomía en la vejez.

\section{Estudio 1: Metodologías}
\subsection{Diseño de la Investigación}
\subsection{Instrumentos y Técnicas}
\subsection{Procedimiento}
\subsection{Análisis de Datos Previos}

%\subsection{Metodologías para Medir la Atención y la Retención del Aprendizaje}
%\subsubsection{Técnicas para Evaluar la Atención en Interacciones Digitales}
%\subsubsection{Instrumentos de Medición de la Retención del Aprendizaje}
%\subsubsection{Herramientas para la Evaluación Emocional en Contextos de Aprendizaje}

\section{Estudio 2: Metodologías }
\subsection{Diseño de la Investigación}
\subsection{Instrumentos y Técnicas}
\subsection{Procedimiento}
\subsection{Análisis de Datos Previos}

%\subsection{ Diseño de Secuencias de Aprendizaje Emocionalmente Optimizadas}
%\subsubsection{Principios de Diseño de Secuencias de Aprendizaje Óptimas}
%\subsubsection{Integración de Emociones en la Secuenciación del Aprendizaje}



