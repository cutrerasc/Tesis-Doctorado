\section{Objetivos de Aprendizaje}
\subsection{Pensión Pública - PGU}
\begin{itemize}
    \item Identificar los requisitos necesarios para acceder a la Pensión Garantizada Universal (PGU), detallando los criterios de edad, residencia y situación económica.
    \begin{itemize}
        \item Entender el requisito de edad para acceder a la PGU y los plazos para la solicitud de esta. 
        \item Conocer las condiciones de residencia en el territorio Chileno para acceder a la PGU
        %considerando un periodo no inferior a 20 años desde que se cumplen 20 años y un periodo no inferior a 4 años en los últimos 5 años inmediatamente anteriores a la fecha de solicitud del beneficio. 
        \item Entender los requisitos de situación económica para poder solicitar la PGU. %PFP
        % Ingresos por otras pensiones (sobrevivencia accidentes de trabajo y sistema antiguo de pensiones
        \item Entender las excepciones para acceder a la PGU. 
        %Pensionados por CAPREDENA y DIPRECA 
        %\item Conocer otros beneficios complementarios a la PGU.%Asignación familiar y bono por hijo son parte de otra estructura, son independientes a la PGU  
        \item Comprender el proceso a seguir para cambio entre pensión básica solidaria por invalidez a PGU. 
        % paso de pensión de invalides a PGU al cumplir los 65 años. 
    \end{itemize}
     \item Conocer cuáles son los montos de PGU a los que pueden acceder según sus características personales %Hay personas que no pueden acceder a la PGU completa, pero si a una parte de ella para complementar su pensión autofinzanciada.  
    \item Conocer el proceso de solicitud de la PGU, incluyendo los documentos necesarios, los pasos a seguir y los lugares donde es posible realizar la solicitud. 
    %\item Comprender las alternativas disponibles para cobrar una Pensión Garantizada Universal (PGU) que no ha sido retirada.
    \item Reconocer las causales que pueden llevar a la extinción de la Pensión Garantizada Universal (PGU), tales como el fallecimiento del beneficiario, el cambio de residencia al extranjero o el incumplimiento de los requisitos de elegibilidad. 

    %\item Como se calcula el PFP (Por revisar) // conocer cual es el umbral de corte para acceder a PGU  
    %\item Comparar la Pensión Garantizada Universal (PGU) con otras modalidades de pensiones en Chile, destacando las diferencias en cobertura, montos y requisitos.
   % \item Comprender el funcionamiento y los fundamentos del Puntaje de Focalización Previsional. 
    %Que se considera para calcularlo, cada cuánto tiempo se calcula, que puede hacer cambiar este puntaje y que hacer en el caso de que los datos con los que se calculó el puntaje no correspondan o requieran ser modificados (RSH por ejemplo) 
\end{itemize}
\newpage
\subsection{Pensiones Privadas - AFP/Aseguradoras}
\begin{enumerate}
    \item Identificar y entender las diferentes modalidades de pensión (como Renta Vitalicia Inmediata, Retiro Programado, Retiro Programado con Renta Vitalicia Diferida y Renta Vitalicia Inmediata con Retiro Programado) que se pueden seleccionar en el formulario SCOMP.
    \begin{enumerate}
        \item Entender en qué consiste la modalidad de Renta Vitalicia Inmediata, incluyendo sus condiciones especiales de cobertura, como periodos garantizados y escalonados.
        \item Entender cómo funciona la modalidad de Retiro Programado, incluyendo los factores que influyen en los montos y sus cambios a lo largo del tiempo.
        \item Entender cómo funciona la modalidad de Renta Temporal con Renta Vitalicia Diferida, incluyendo el periodo de diferimiento, y las condiciones que afectan los montos de pensión.  
        \item Entender la modalidad combinada de Renta Vitalicia Inmediata con Retiro Programado y las implicancias para el pensionado %Muy poco usada, más secundario de entender 
    \end{enumerate}
    
    \item Entender cómo las características individuales o de grupo familiar influyen sobre las preferencias de elegir alguna de las distintas modalidades de pensión. (Beneficiarios de sobrevivencia y herencia)
    \item Comprender las principales diferencias entre las modalidades de pensión. 
    \item Entender los conceptos que se utilizan al momento de rellenar el SCOMP. 
    \item  Conocer el proceso completo de la solicitud de pensiones.
    \item Entender las principales causales que generan incertidumbre o molestia (Ejemplo: ¿Qué pasa si los fondos de AFP son muy pequeños y se acaban?)
    \item Entender elementos complementarios para la toma de decisiones.
    \begin{enumerate}

        \item Excedentes de libre disposición 
        \item Ahorro previsional Voluntario APV
    \end{enumerate}
    
    %, bono por hijo(público)) 
    %\item Entender cuáles son las implicancias de la composición del grupo familiar en el proceso de toma de decisiones.  juntar con el de arriba 
    %\item Entender cuales son las implicancias de la mortalidad para el grupo familiar.  
    
\end{enumerate}

%\section{Recursos de Aprendizaje}
%\subsection{Pensión Pública - PGU}
%\begin{enumerate}
% %    \item Guía Interactiva en Línea
%     \begin{itemize}
%         \item \textbf{Descripción:} Una guía interactiva en línea que permita a las personas explorar cada uno de los requisitos de la PGU de manera detallada, con secciones dedicadas a la edad, residencia, situación económica, excepciones y beneficios complementarios.
%         \item \textbf{Utilidad:} Este recurso sería útil porque facilita la autoexploración y el aprendizaje autodirigido, permitiendo a las personas navegar a su propio ritmo y enfocarse en las áreas que necesitan comprender mejor. Además, la interactividad aumenta el compromiso y la retención de la información.
%     \end{itemize}
%     \item Infografías y Mapas Conceptuales
%     \begin{itemize}
%         \item \textbf{Descripción:} Conjuntos de infografías y mapas conceptuales que visualicen los requisitos para la PGU, el proceso de solicitud, las alternativas de cobro y las causales de extinción.
%         \item  \textbf{Utilidad:} Estos recursos visuales serían útiles para resumir y simplificar la información compleja, haciéndola más accesible para los usuarios, especialmente para aquellos que prefieren aprender visualmente. Las infografías pueden ser descargadas o impresas, lo que permite revisarlas en cualquier momento.
%     \end{itemize}
%     \item Videos Explicativos
%     \begin{itemize}
%         \item \textbf{Descripción:} Una serie de videos breves que expliquen cada uno de los aspectos mencionados: requisitos para acceder a la PGU, proceso de solicitud, alternativas de cobro, y causales de extinción.
%         \item \textbf{Utilidad:} Los videos son útiles para proporcionar explicaciones claras y concisas, ya que pueden estar acompañadas de ejemplos y narrativas que pueden ayudar a los usuarios a comprender mejor los conceptos. También permiten pausar y revisar la información según sea necesario.
%     \end{itemize}
%     \item Simulador en Línea
%     \begin{itemize}
%         \item \textbf{Descripción:} Un simulador en línea donde los usuarios puedan ingresar diferentes variables (como edad, residencia, y situación económica) para ver si califican para la PGU, cuáles son los pasos a seguir para la solicitud, y qué sucedería en caso de no cumplir con los requisitos.
%         \item \textbf{Utilidad:} Este recurso práctico sería muy útil para aplicar los conocimientos de manera simulada, lo que ayudaría a reforzar la comprensión y hacer tangible la información aprendida. El simulador también permite a los usuarios experimentar con diferentes escenarios sin consecuencias reales, lo que facilita el aprendizaje.
%     \end{itemize}
% \end{enumerate}

% \subsection{Pensiones Privadas - AFP/Aseguradoras}


