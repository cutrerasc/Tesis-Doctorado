\documentclass[12pt,a4paper]{article}
\usepackage[spanish]{babel}
\usepackage[utf8]{inputenc}
\usepackage[T1]{fontenc}
\usepackage{lmodern}
\usepackage{setspace}
\usepackage{geometry}
\usepackage{graphicx}
\usepackage{hyperref}
\usepackage{fancyhdr}

\geometry{margin=2.5cm}
\onehalfspacing

% Configuración del encabezado y pie de página
\pagestyle{fancy}
\fancyhf{} % Limpia los encabezados y pies de página
\renewcommand{\headrulewidth}{0.4pt} % Línea bajo el encabezado
\renewcommand{\footrulewidth}{0pt} % Sin línea sobre el pie de página

% Definición del encabezado
\fancyhead[L]{\includegraphics[height=1.5cm]{logo_universidad.png}}
\fancyhead[C]{\small{
    \begin{tabular}{c}
        \textbf{Universidad [Nombre de la Universidad]} \\
        Facultad de [Nombre de la Facultad] \\
        Departamento de [Nombre del Departamento]
    \end{tabular}
}}
\fancyhead[R]{}

% Definición del pie de página
\fancyfoot[L]{Proyecto de Tesis Doctoral}
\fancyfoot[R]{\thepage}

\begin{document}

% Título y datos del estudiante
\begin{center}
    \vspace*{2cm}
    
    {\LARGE \textbf{Título del Proyecto de Tesis Doctoral}}
    
    \vspace{1.5cm}
    
    \textbf{Nombre del Estudiante}
    
    \vspace{1cm}
    
    Proyecto de tesis doctoral presentado para la obtención del grado de Doctor en [Campo de Estudio]
    
    \vspace{2cm}
    
    [Mes, Año]
    
    \vspace{2cm}
\end{center}

\thispagestyle{fancy} % Aplica el estilo 'fancy' a la primera página

% Índice
\tableofcontents
\newpage

\section{Introducción}

Presenta una visión general del tema de investigación, su relevancia y contexto.

\section{Justificación}

Explica la importancia del estudio y por qué es necesario investigar este tema.

\section{Objetivos}

\subsection{Objetivo General}

Enuncia el objetivo principal de la investigación.

\subsection{Objetivos Específicos}

\begin{itemize}
    \item Objetivo específico 1
    \item Objetivo específico 2
    \item Objetivo específico 3
\end{itemize}

\section{Marco Teórico}

Desarrolla las teorías y conceptos fundamentales relacionados con el tema.

\section{Metodología}

Describe el diseño de investigación, métodos, técnicas y procedimientos que se utilizarán.

\section{Cronograma}

Presenta un plan de trabajo detallado con las etapas y tiempos estimados.

\section{Referencias}

\bibliographystyle{apalike}
\bibliography{bibliografia}

\end{document}
